\documentclass[11pt,a4paper]{letter}
\usepackage[top=1.00in, bottom=1.0in, left=1.1in, right=1.1in]{geometry}
\usepackage{graphicx}

%\signature{}

\begin{document}
\begin{letter}{}
\includegraphics[width=1.1\textwidth]{/Users/Lizzie/Documents/Professional/NCEAS/Manuscripts/_letterheads/brchead.png}
\opening{Dear Dr. Cushman:}

Please consider our manuscript `Phenological niches and the future of invaded ecosystems with climate change' for consideration as an invited review in \emph{AoB PLANTS}.

Increasing research in invasion biology has linked phenology to exotic plant species across the globe, while at the same time climate change research has documented dramatic shifts in plant phenology. Both these fields have provided tremendous new data and systems to study further, but both have also tended to focus more on documenting effects, compared to linking research back to basic biological questions. Here, we outline how increasing interest in examining how phenology may affect plant invasions could be re-approached from a more focused framework---one that considers phenology as one component within the constant plant battle to optimally balance acquisition, allocation and loss in a globe where most systems' variable climates are now also highly non-stationary. Working across a growing season from early, middle to late we review how phenology may affect how plants avoid and moderate how plants experience disturbance and stress---via climate, herbivory and competition---throughout the growing season. We detail how work on plant invasions and phenology within this framework would provide a more rigorous test of what drives invader success, while at the same time testing basic plant ecology theory. Additionally, we discuss how extensions of this framework could begin to model how ecosystems themselves may shift in the future with continued climate change. 

We suggest the following possible reviewers:
\begin{itemize}
\item Richard Primack, Boston University, primack@bu.edu
\item David Inouye, University of Maryland, inouye@umd.edu
\item Abraham Miller-Rushing, National Park Service, abe\_miller-rushing@nps.gov
\end{itemize}

The main manuscript contains 5990 words in the main text, 206 words in the abstract, and two Figures. All figures are designed to be published in color. \\
\\
This manuscript is not under consideration elsewhere. Both authors approved of this version for submission. Thank you for your consideration of this manuscript.

\noindent Sincerely,  \\
\\
 \includegraphics[width=0.4\textwidth]{/Users/Lizzie/Documents/Professional/Vitas/Signatures/SignatureLizzieSm.png} 
\\
\noindent Elizabeth M Wolkovich
\end{letter}
\end{document}
