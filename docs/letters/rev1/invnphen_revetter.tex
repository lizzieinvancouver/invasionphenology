\documentclass[11pt,a4paper]{letter}
\usepackage[top=1.00in, bottom=1.0in, left=1.1in, right=1.1in]{geometry}
\renewcommand{\baselinestretch}{1.1}
\usepackage{sectsty,setspace,natbib,wasysym} 
\usepackage{graphicx}
\usepackage{color}
\usepackage{epstopdf}

\usepackage{fancyhdr}
\pagestyle{fancy}
\fancyhf{} 
\fancyhead[LO]{November 2013}
\fancyhead[RO]{Revision letter}

\begin{document}
\thispagestyle{empty}
\begin{letter}{}
\renewcommand{\refname}{\CHead{Literature Cited}}
\bibliographystyle{/Users/Lizzie/Documents/EndnoteRelated/Bibtex/styles/amnat}
%\includegraphics[width=1.1\textwidth]{/Users/Lizzie/Documents/Professional/NCEAS/Manuscripts/_letterheads/brchead.png}

\opening{Dear Dr. Dukes:}

Please consider our manuscript `Phenological niches and the future of invaded ecosystems with climate change' for consideration as a review in \emph{AoB PLANTS}. This is a revision of manuscript \#13061.\\
\\
Two referees and you provided very helpful and supportive comments on the previously submitted manuscript. We cannot extend enough thanks for the positive tone of the comments and the very useful ideas and perspectives included in them. In response we have completely overhauled the manuscript and hope you and the referees will find the current version pleasantly distinct from our first submission. \\
\\
We have adjusted the aim of the paper, making it clear that our goal is to update and extend basic hypotheses regarding phenology and plant invasions to include plant strategies and, relatedly, plant functional traits, and extend these to predictions with climate change; we have provided a clear roadmap of this in the abstract and introduction now. To this new aim, we have added a literature review of phenology and its relationships with other functional traits (Table 1). We then have set up clear assumptions leading to explicit predictions of when and by which mechanisms plant species may invade three distinct ecosystem types based on climate change scenarios. A following section considers current evidence for these predictions and we close by laying out major questions in the field. Finally, we have greatly increased the number and diversity of our citations, taking care to add citations beyond recent work by Wolkovich and colleagues wherever possible. We detail these extensive changes below.\\
\\
The main manuscript contains 5,885 words in the main text, 275 words in the abstract, one table and three figures. All figures are designed to be published in color. \\
\\
This manuscript is not under consideration elsewhere, and both authors approved of this version for submission.\\
\\
Thank you for your consideration of this manuscript.\\
\\
Sincerely,
\\
 % \includegraphics[width=0.4\textwidth]{/Users/Lizzie/Documents/Professional/Vitas/Signatures/SignatureLizzieSm.png} 
\\
\noindent the authors

\newpage
Below we give all original referee comments and our responses to each
review under the `Editor' or `Referee' subheadings. Comments are given in \emph{italics} while
our responses are in regular print. 

\noindent {\bf Editor:}\\
\\
\emph{I think that both reviewers can see nuggets of possibility in this manuscript, but both felt fairly strongly that in its current state it lacks sufficient structure and novelty. Reviewer 1 has provided some nice suggestions for how to revise (and reconceptualize) the paper in a way that would produce a more novel, compelling, useful, and comprehensive product. So, while I suspect the initial feedback on the manuscript may be disappointing, I think the collective suggestions of both reviewers provide a pretty nice roadmap for some ways that you can develop the manuscript into a more compelling package, and one that is better differentiated from your previous work. Please consider all reviewer comments in your resubmission. }\\ 

We thank Dr. Dukes for remaining so positive and upbeat on our behalf in the face of rather difficult but generally deserved feedback. We greatly appreciate his time and help in improving this manuscript and hope that he and the referees will find it very much improved.

\noindent {\bf Referee 1}\\
\\
\emph{This is an interesting and exciting paper that works to better integrate phenology into standard ecological theory, particularly major plant strategies and life history strategies, including traits, in a multi-stressor context. This is a conceptual piece, potentially very useful for placing understanding of phenology into plant ecological theory, through the 'lens' of phenology, invasions and change. For example, although phenology is critical to the concept of temporal niche differentiation and plant strategies such as Grime's CSR (or maybe Tilman's model), it is seldom considered as such. Thus, there is great potential to better integrate phenology into existing (aka contemporary) plant ecological theory.}\\

Special thanks to referee 1 for his/her positive attitude given our initial submission. We hope we have been able to address many of his/her concerns in our new submission and have greatly appreciated from his/her suggestions. \\ 

\emph{However, I'm not sure that (as written) this manuscript is as effective as it could be to meet that goal. There are several overarching issues that I'll try to describe. \\ First, it's not clear how this paper is necessarily a conceptual advance on the Wolkovich and Cleland 2011 paper in Frontiers Ecol and Environment. In fact, there are numerous references to that paper, which was quite comprehensive and very excellently written, but there is not a clear differentiation between the two pieces of work. This could be alleviated, perhaps, by an explicit description of differences between the two (as well as the new Wolkovich paper in AJB, given the number of citations to that paper; see also comment below).}\\ 

This is a good point. \cite{wolkovich:2010fee} laid out basic hypotheses for linking phenology to invasions and \cite{wolkovichAmBot2013} tested several of these hypotheses. This manuscript, in contrast, emphasizes more direct mechanisms and links between phenology and other traits---links that could both help better understand and predict connections between phenology and invasions. We now lay this difference out more clearly in the abstract and lines 17-45.
 \\

\emph{Second, the manuscript tends to be relatively unfocused, and in short, could benefit from a "roadmap." The Introduction (lines 1-43) is short, but the last two paragraphs could be more tightly written to better define the goals of this paper. Early portions of the manuscript describe phenology within the context of Grime's competition, stress and disturbance (CSD) model, but this emphasis seems short-lived, as the concept is not further discussed after about page 5 (nor in the conclusions other than passingly). The paragraph on lines 93-103 suggests the need for a model, with testable predictions, but neither seem to be provided. The paper might be stronger if it did develop a model, with testable predictions, then used the literature to `test'' those predictions.\\ Then, the emphasis of the paper shifts to early, mid- and late-season strategies within the context of environmental variation, including extreme events; the link to CSD, while mentioned occasionally, is somewhat lost through this 5.5-page section. The differentiation between mesic systems (does this mean forests?) and temperate grasslands is not clear; technically, I think grasslands are considered mesic. Although this section has lots of great synthesis, perhaps it could be subdivided into subsections that provide a more clear 'roadmap' of how the pieces fit together. } \\ 

We agree that the first submission would have benefitted greatly from a roadmap and have provided one in this submission, while also completely overhauling the manuscript. As we explain in the abstract and intro lines 36-45:

\begin{quote}
Here we build on current theoretical perspectives and empirical work to develop predictions of how phenology may enhance plant invasions with climate change. We first review the role of phenology in avoiding or mitigating various flavors of disturbance, stress, and competition. Next, we review recent literature on plant functional traits to highlight recent evidence for a fundamental trade-off between flowering phenology and the return rate of growth investments, which may in turn impact how climate change and phenology affect invasions. Considering projected scenarios of climate change we make mechanistic predictions for when during the growing season across three major ecosystem types vacant phenological niche space may promote invasions, and consider current evidence for our predictions. We close by reviewing major questions whose answers would improve predictions of future invasions via phenology.
\end{quote}

\emph{Third, although there are many citations, there do seem to be some important papers missing: the Gu BioScience paper in line 177, work by Friedman on the competing role of temperature and photoperiod on spring vs autumn phenology for line 219-229, prior work on autumn phenology by Richardson and colleagues for line 254, the paper by Lew Ziska on frost controls on invasive ragweed across the Great Plains at line 263, Miller-Rushing and Inouye, and maybe Loik, at line 286, autumn rainfall effects on cheatgrass establishment at line 298, Dukes and Mooney line 443, etc. In contrast, the surprising findings of Fridley 2012 receive substantial description, and prior work by Wolkovich and Craine seems disproportionately cited.}\\ 

We appreciate the referee's concern that several studies seemed overcited and his/her concern that we cite additional work. We now cite \cite{gu2008} and recent work by \cite{dukes2011}. We have additionally added a suite of new citations throughout the paper, including additional citations in the late season section. We did not include all of the suggested references for varying reasons related to clarity. For example, \cite{ziska2011} is on ragweed in North America, which is a native species (but often considered invasive) to North America. While we understand that native species can also be invasive a discussion of this is outside the scope of the paper and most of our assumptions and hypotheses are clearly related to exotic and exotic invasive species, thus we have decided not to incude this citation. We note that in the current submission there are over 110 citations and only 14 are by Craine, Fridley, or Wolkovich.\\ 

\emph{Fourth, there is an interesting section on plant traits starting on line 381. There is good potential here to actually define phenological traits (as opposed to simple phenology as a trait); this may beyond the scope of this paper, but would be a very nice contribution (e.g., allowing phenophases? or sensitivities to be compared to other traits). There is some talk about including phenology in BIEN, but it doesn't seem to be there yet. If traits could be defined, we could populate that table in BIEN and really start to get at that ``holistic trait framework'' described on line 401. Also, a fair amount of work has been done with early and late-flowering traits (?) of Arabidopsis. Such an approach may actually help integrate the 4 paragraphs starting on page 381. (A sub-header here would help, too.) } \\ 

We greatly appreciate these comments and agree with the referee that we missed an oppotunity to develop phenology within the functional traits literature in our first submission. A full, or even partial, discussion of the suite of traits that phenology encompasses is outside the scope of this paper and furthermore, should probably be taken on by an international team of collaborators. Even a quick glance at the efforts behind the phenophases of the COST725/PEP725 databases suggests a tremendous amount of work here. 

Thus we have instead chosen to look more closely at what the current traits literature suggests correlates strongly with phenology. We present a brief literature review in Table 1 now and build on it in lines 75-109 as well as variously throughout the manuscript.\\

\emph{The paragraph starting on line 195 seems a bit out of place with a focus on precipitation. }\\

We agree and as part of the revision this paragraph has been completely shifted and adjusted.\\

\emph{The value (or constraints) of AGDD models could be discussed in the paragraph starting on line 326.}\\

Given the new scope of the paper, and word limits, we could not include a discussion of this sort.\\ 

\emph{The concept about low-latitude origins starting on line 340 is quite interesting, and could be discussed in additional detail. Similarly, it might be interesting to consider successional state as a control on phenological plasticity. }\\

We agree that these are both very interesting topics. Unfortunately, given the new scope of the paper, and word limits, we have not included a discussion of these topics in our current submission. \\

\emph{Did Cleland et al. 2012 look at exotics (line 405)? (I looked again, and had trouble finding the distinction between exotics (invasives) and native species that would support this line.) }\\ 

Apologies for this error. The Cleland et al. citation only refers to the second part of the sentence. We have now adjusted it (lines 368-371) to read: 
\begin{quote}
Recent work shows that temperate exotic species are more phenologically responsive to temperature than native species \citep{Willis:2010al,wolkovichAmBot2013} and that species that advance with warming also tend to increase in abundance and performance \citep{Cleland:2012vn}. 
\end{quote}

\emph{Line 418: ...unless flowers come before leaves... }\\

We have deleted this section of the manuscript during the revision process.\\

\emph{Line 421: Are phenological events necessarily coupled to one another in Predictable sequence? }\\

We have deleted this section of the manuscript during the revision process.\\

\emph{Last sentence of the paper, line 454. Would it be possible to actually transcend from considering climate change x invasives x phenology to climate change x species (performance?) x phenology (as in Cleland et al. 2012). This might enable us to make broader generalizations about Plants and their interactions (e.g., Hulme's work), which would meet one of the goals of this paper, to create a deeper understanding of plant phenology within the context of traditional plant ecological theory. As it stands, the line is completely consistent with the title, though much of the rest of the paper takes the broader approach of considering all plants as opposed to only "invasives." }\\

This is an excellent point and we hope the referee will find our current submission more inline with considering phenology and its role in plant performance. Our main hypotheses are, however, often based on species that have not adapted to a system's long-term climate means---thus we focus much on exotic species---and have chosen to maintain this focus. \\

\emph{Figure 2 is nicely constructed, but it's not clear how it contributes to the thesis of the paper.}\\

We use this figure to highlight differences in systems with dominant temperature control versus those with temperature and precipitation control. The first version of this figure was, however, confusing and contained too many ideas. In this version we now include native versus exotic flowering curves, we have limited the number of factors we shows in boxes below the graphs---to highlight drought differences. We also use the figure often to show commonalities and differences between these systems in their climate and how certain climate variables are linked (e.g., standard deviation and mean temperatures). We believe the current version of this manuscript makes the utility of this figure much clearer. \\

\emph{There are a few typos here and there.}\\

We apologize for this and have done our best to address typos in this submission.\\

\noindent {\bf Referee 2}\\
\\
\emph{This essay review on plant phenology in the context of species invasions and climate change aims to provide a new holistic framework of phenology research based on plant strategy theory. The basic idea is presented as Fig. 1, which highlights seasonal niches and varying constraints on plant performance (presumably fitness), ranging from high stress and disturbance for early and late phenology species and high competition for those species in the middle. It is argued that much recent work has fixated on phenology as a single trait conferring (for example) fitness advantages to invaders (e.g., p. 15), and the authors advocate instead a more integrated approach that also considers herbivores, other physiological constraints (e.g. leaf economics spectrum), etc. \\ Simply put, I found this paper to offer little in terms of novelty, to be overly derivative of other recent work, and lacking in delivery of its basic premise of a new framework of phenology built on plant strategies. There is no real framework here, only a review of a few other (mostly recent) studies that have found flowering or foliar phenology to be linked to other plant traits. It has been appreciated since the earliest treatments of plant strategy 'theory' (even before it was such) that phenology is a core component of various strategies that reflect how plants have evolved in response to stress, disturbance, and competition: the authors may wish to consult Al-Mufti et al (1977 JEcol) as an early comprehensive treatment of the subject for herbaceous vegetation, and reflect on the central message of Lechowicz's masterful 1984 Am Nat paper (cited by the authors) that tree foliar phenology is one of many complex and interrelated traits dictating success in a given environment.\\
These are old papers and ideas; rather than set up a strawman that the phenology literature ignores or is uninterested in integrated plant function, it may be more profitable to specifically revisit constraints on phenology previously identified (e.g., Grime and Mowforth's argument about early phenology grassland species and genome size, not mentioned here). }\\ 

We appreciate the referee's concerns and agree that our previous submission did not provide enough in the way of a novel framework. We have worked hard in this submission to address these concerns. Further, we completely agree with the referee that phenology has long been considered within plant strategy theories---our aim here was to link some of this literature to recent work, which has often focused narrowly on phenology. Additionally, while previous work has considered phenology as an important functional trait in plant strategy theory, there is certainly a variety of views on this \citep[e.g.,][]{Ollerton:1992kf} and recent phylogenetic work further suggests renewed consideration of phenology as one important trait could benefit much current research.

We appreciate the referee's suggestion of citations. We now cite \cite{Almufti1977}. We have chosen not cite Grime's work on genome size \citep{Grime:1982xv}, as a full discussion of the current state of this literature was beyond the scope of the paper \citep[for example, see][]{Bennett1995,gregory2001}.\\

\emph{To the extent there is any conceptual synthesis here, it is not clear that it advances this subject conceptually beyond other recent reviews-for example, Fig. 1 here is closely allied to Fig. 1 of Wolkovich \& Cleland 2011 (Frontiers)-and there is much retelling of very recent work by what I assume are the authors (much of the narrative is driven by a few citations, such as Wolkovich et al. 2013). Much of the rest of the paper is of the form "we need more data about X"; most of these recommendations are obvious and seem unlikely to change how current data collection and analyses are being performed. \\ In sum, I disagree with the basic premise of this paper that phenology's role as an integrated behavioral trait is being ignored, and did not find much otherwise of suitable novelty. }\\

We are sorry to hear the referee was so disappointed in the first version of this manuscript and believe this version is much improved over its predecessor. We now lay out a clearer roadmap for the paper and what differentiates it from other work. As mentioned in our response to referee 1, lines 36-45 include a roadmap and the introduction overall includes a stronger distinction between previous work and this manuscript.

Finally, we have completely cut the section on future data needs and shifted it into a series of major questions we believe are most critical to advancing the field.

\newpage
\bibliography{/Users/Lizzie/Documents/EndnoteRelated/Bibtex/LizzieMainMinimal}

\end{letter}
\end{document}
