\documentclass[11pt,a4paper]{letter}
\usepackage[top=1.00in, bottom=1.0in, left=1.1in, right=1.1in]{geometry}
\renewcommand{\baselinestretch}{1.1}
\usepackage{setspace,natbib,wasysym} 
\usepackage{graphicx}
\usepackage{color}
\usepackage{epstopdf}

\usepackage{fancyhdr}
\pagestyle{fancy}
\fancyhf{} 
\fancyhead[LO]{March 2014}
\fancyhead[RO]{Revision letter}

\begin{document}
\thispagestyle{empty}
\begin{letter}{}
\includegraphics[width=0.4\textwidth]{/Users/Lizzie/Documents/Professional/images/letterhead/arnold/AASmLogo2colr.jpg}
\pagenumbering{gobble}
\renewcommand{\refname}{\CHead{Literature Cited}}
\bibliographystyle{/Users/Lizzie/Documents/EndnoteRelated/Bibtex/styles/amnat}

\opening{Dear Dr. Dukes:}

Please consider our manuscript `Phenological niches and the future of invaded ecosystems with climate change' for consideration as a review in \emph{AoB PLANTS}. This is a second revision of manuscript \#13061.\\
\\
Two referees and you provided helpful comments on the previously submitted manuscript. We have worked to address all comments and detail our responses point-by-point below.\\
\\
The main manuscript contains 5,994 words in the main text, 267 words in the abstract, one table and three figures. All figures are designed to be published in color. \\
\\
This manuscript is not under consideration elsewhere, and both authors approved of this version for submission.\\
\\
Thank you for your consideration of this manuscript.\\
\\
Sincerely,
\\
\includegraphics[width=0.4\textwidth]{/Users/Lizzie/Documents/Professional/Vitas/Signatures/SignatureLizzieSm.png} 
\\
\noindent Elizabeth M Wolkovich

\newpage
Below we give all original referee comments and our responses to each
review under the `Editor' or `Referee' subheadings. Comments are given in \emph{italics} while
our responses are in regular print. 

\noindent {\bf Editor:}\\
\\
\emph{I think this is a useful review of important concepts linking invasions, phenology, and climate change. Both reviewers have some concerns that should be addressed, and reviewer \#2 is particularly critical, but I feel that the manuscript will make a nice contribution, provided most of the reviewers' suggestions are addressed.}

We thank Dr. Dukes for his supportive and helpful comments. We hope we have been able to address all of his and the reviewers' concerns.\\

\emph{One overarching recommendation is that you carefully consider each instance of the word ``exotic'' and be sure that you don't really mean ``invasive,'' or you aren't referring to differences that are specific to a single location / continent (e.g., see lines 361-362). If you are referring to differences in a single location, specify that. In my view, there is no reason an "exotic" species should necessarily differ in traits from a native -- as you know, that term only implies that the species comes from another continent, so an exotic species on one continent is native on another. (You might also consider using ``non-native'' instead of exotic).}

We appreciate Dr. Duke's concern over our terminology, as this is obviously a sticky area for the invasion biology literature in general. We may have divergent views, however, because we believe that trait differences may arise from the fact that non-native species do not have a long-term history with the introduced system's species and climatic conditions, in contrast with native species. Further, because many invasive species have long lag times we see the distinction between `non-native' and `invasive' as a difficult one. We have, however, gone through and adjusted wherever necessary our terminology (lines 25-26, 32, 159, 286-287). Additionally, we are more upfront with our definitions. In lines 46-54 we now state:
\begin{quote}
\noindent \emph{A note on terminology}\\
Given debate over terminology in invasion biology \citep{Colautti2009} we wish to be clear about our definitions. We use `non-native' to refer to any species established outside of its home range; such a distinction between native and non-native is important because non-native species have evolved in a different community than the one into which they have been introduced. Thus, they may exhibit differing strategies and trade-offs than native species.  We use the term `invasive' for non-native species with a detrimental impact in their introduced community \citep[following][]{Mack:2000bn}. Finally, `invade' and `invasion' refer to the introduction of species, whether they are eventually invasive or not.\\
\end{quote}
Finally, we have switched all instances of our use of `exotic' to `non-native.'\\

\emph{One other point -- this ms considers three types of systems. Are there any easy lessons to extrapolate to other systems? Might be worth a little discussion somewhere.}

This is a good point and various drafts have included or excised a section related to this (shown directly below), based on our word limit. We attempted to include it in this draft but being far above our word limit we decided in the end to cut it. \\

\begin{quote}
\noindent \emph{Extensions to additional systems}\\
\noindent We have focused above on systems generally with one pronounced growing season, yet the opportunity for phenology to impact plant invasions clearly extends to other systems. In particular systems with variability in when the start of season occurs and in the related magnitude of the climate event that starts the season (e.g., mm of first major rainfall in many desert systems) may be particularly prone to early season invasions, especially if climate change has shifted climate dynamics. Phenological enemy release should also be considered as a possible mechanism for invader success in any system with variation in herbivory pressure between or during seasons. For example, an early-season enemy release mechanism may also extend to tropical systems with a pronounced wet-season, where both plant and herbivore phenology may be cued by rainfall. Research using 32 species of tree saplings in Panama showed variation across the season in herbivore pressure (lower during the dry season) and reduced herbivory on species that flushed synchronously during the wet season \citep{aide1993}.  Although untested, these results suggest that tropical non-native species could benefit from phenological enemy-release.
\end{quote}

As noted in lines 125-126, we do consider these predictions general and try to state them this way. We are, however, out of space for a longer discussion of this (such as the one shown above).\\

\emph{Additional minor comments are below. Let me also apologize for holding your manuscript up for longer than I intended -- with the referees' very different opinions, I wanted to find time to do a thorough read myself, and it took longer to make time for it than I had hoped. I'm sorry!}

We (especially Lizzie, who is the complete culprit) are also sorry for how long our revisions took. It is impressive how something like the term `faculty' can swallow whole so much free time.\\

\emph{L 65: Delete "While" and see reviewer comments re biotic.}

Done, this sentence now reads `In contrast, abiotic constraints on plant growth (e.g. frost) are unlikely to be impacted by plant phenology.'\\

\emph{L 71 suggests}

Done.\\

\emph{L 168-171 Awkward phrasing}

Agreed, we have changed this into:

\begin{quote}
Recent increases in spring temperatures in the temperate biome \citep[at least partially associated with increases in greenhouse gases, see][]{Trenberth:2007hk} have been studied extensively.
\end{quote}

\emph{L 178 Exotic species often more plastic: Citation?}

We have added three citations here (see text quote below).\\

\emph{L 179 ``Adding in trait correlations with ...'' Confusing. Rephrase?} 

Agreed, this section now reads:
\begin{quote}
Next, if non-native species have higher phenological plasticity \citep[e.g.,][]{wainwright2013} they may more closely track shifting climate than native species. Early-flowering is also often correlated with plant traits related to rapid growth. 
\end{quote}

\emph{L 202 ``...drought'' Citation?}

We have added two citations.\\

\emph{L 203 delete ``species'''}

Done, thanks for catching this!\\

\emph{L 205-6 Seems speculative. Acknowledge this or strengthen with evidence?}

It is speculative, so we know acknowledge that (lines 210-212)\\

\emph{L 209 season} 

Done.\\

\emph{L 253 ``will need to'' Why? Describe the benefit instead of saying this?}

Good point, we have adjusted the manuscript as follows:
\begin{quote}
Moving forward, accurate predictions of phenologically-mediated invasions will require teasing out exact mechanisms. Thus, future research with climate change in temperate biomes will need to quantify how much invasion success occurs because of flexibility in phenology ....
\end{quote}

\emph{L 269 suggest deleting ``highly''}

Done and thank you.\\

\emph{L 330 Topic sentence does not seem to lead well into paragraph. The paragraph is not about extreme events. Also, this paragraph seems a bit out of place.}

This is a good point. We think the topic sentence is actually accurate but the rest of the text did not well connect to our (well-hidden!) meaning. We have thus adjusted as follows:

\begin{quote}
A coherent picture of how species respond to shifts in temperatures and precipitation will include a focus not just on mean or aggregated climate metrics, but also on extreme events. An example of the importance of this comes from attempts to understand the correlation between phenology and performance with climate change. Several studies show that species that tend to advance with warming also tend to increase in abundance or performance \citep{Cleland:2012vn}, including invasive species \citep{Willis:2010al,chuine2013}; yet other work shows that the early season (native) species most sensitive to climate are those that suffer the greatest performance losses with climate change \citep{Inouye:2008gj}. Such conflicting results can be better understood when considering the role of extreme events. In the latter study, a shift in earlier springs that did not coincide with a shift in frost dates produced the performance declines \citep{Inouye:2008gj}. 
\end{quote}

\emph{L 349 Could this paragraph be condensed?}

Good point, there was extra text here that could be cut and room to adjust for clarity. The paragraph now reads:
\begin{quote}
Studies of precipitation-controlled systems must also deal with large-scale, longer-term cycles in precipitation that often dominate in such systems (e.g., El Ni\~{n}o in many semi-arid systems in western North America). Such longer-term cycles may be a critical consideration because native species may be adapted to these cycles. Relevant studies of population dynamics will, therefore, need to work across the relevant climate oscillation timescales and predictions will need to consider whether the oscillation may shift with climate change, as projected \citep{ipccPhys2007}. Such oscillations may also be directly important to exotic species as they may dictate invasion lags and jumps \citep{Salo:2005eo}. \\
\end{quote}

\emph{L 368 Clarify this is in some sites, not everywhere?\\
{L 392 adjust to}}

Both done.\\

\emph{L 416 explain why they "need to" earlier?}

This is a good point and we tried several avenues to adjust this and move up the `why.' However, all attempts seemed clunky. Thus we have adjusted the `need to' sentence and started a new paragraph (starting on line 427) that more clearly addresses `why' in the topic sentence and then reviews the reasoning in detail throughout the rest of the paragraph.\\

\emph{L 421-430 Might be worth acknowledging that this point can be important from an academic standpoint, but may be irrelevant when it comes to practical matters / management / impact concerns.}

We respectfully disagree that this point is mostly academic: phylogenetic structure is important in management in that if phlyogeny, and not the trait under investigation, is the best predictor of invasion then regulations of which species to allow into countries should be based on phylogeny, in our opinion. We attempted to add some of this to the text but felt that (1) the related section was already too long in comparison to other sections and (2) it was outside the scope of the paper. \\

\emph{L 438 correlates with}

Done, thanks.\\

\emph{Fig 3 temperate grassland row of table: net effect } 

Done, thanks.\\

\noindent {\bf Referee 1}\\
\\
\emph{Overall, this manuscript is much improved from the original version, and it better serves as a state of knowledge regarding interactions among invasions, phenology, climate change and species traits. The manuscript has a better flow, mixing case studies, predictions and tests of those predictions, conceptual and data-driven figures, and synoptical tables.\\
In a few places, there are some redundancies; for example, the Willis et al 2010 paper is mentioned several times with relatively general prose. The section on shifts in climatology is somewhat weak; there are indeed nice analyses of shifts in timing of late-season frosts (see, for example, work by Easterling).}

We appreciate the referee's positive comments. 

We reviewed our use of the Willis paper throughout the manuscript---a total of six times---in all cases it is cited alongside other citations; in two cases we felt we could remove the Willis citation and have. We have added a citation to the Easterling paper, alongside several other more recent studies we already had included:

\begin{quote}
To date, increases in frost risk with spring warming have been documented in parts North America \citep{Inouye:2008gj,gu2008,Augspurger2013}, but not much of North America \citep{Easterling:2000sa}, nor in Europe \citep{Menzel2003a,Scheifinger2003} or China \citep{Dai2013}, where shifts in freeze and frost days have occurred in step with the climate shifts driving earlier spring onset \citep{Dai2013}. 
\end{quote}

\emph{Potential applications for phenological shifts and invasions in changing environments are not addressed. For example, resource managers need information to inform decision-making: which species are likely to become more invasive through phenological change?, or which systems are most susceptible to phenologically mediated invasions? Can we use plasticity of invasion to inform vulnerability assessments? When is the best time to detect in invasion, and where, etc.?\\
However, as written this article provides a nice synopsis of our science understanding of phenology in changing environments, and provides an analysis of potential mechanisms that might explain observed patterns of phenologically mediated invasions.}

This is a good point. We have addressed management implications more in previous work \citep[see][]{wolkovich:2010fee}, but here have focused more on the scientific questions as we believe there is much nuance that needs consideration (and we are close to our word limit). We have added at the end of the paper, however, a new version of the conclusions, which includes:
\begin{quote}
As increasing research builds to test and advance this framework, resource managers will in turn need to evaluate how they may use phenology in their decision-making. If many species appear to gain a foothold or spread in introduced communities via phenology it may suggest novel management practices including which species may have a high potential to be invaders with climate change into certain systems and when the best time to apply treatments may be.
\end{quote}

\emph{Sentence on line 23-28. This is a synopsis of the 2011 manuscript, so perhaps the citation should sit at the end of line 28?}

This is a good point, we have moved the citation down as suggested.\\

\emph{Line 65: Partial sentence, and should that read `abiotic' instead of `biotic'}

Yikes! Thank you for catching this (and to Dr. Dukes as well). We changed to `abiotic' and removed `while' to correct the sentence.\\

\emph{Line 111: The term ``phenological invasion'' is used in the header; this is an interesting idea, and though it implies a mechanism, I'm not sure it works in parallel to biological invasion... Moreover, the term proper doesn't get addressed or explained in the following paragraph. So, I'd suggest the authors consider the necessity of introducing a new term in this case...}

We changed this to `climate change, phenology, and invasions.'\\

\emph{Line 112-113: I'd suggest provision of a citation here for people unfamiliar with stationarity...perhaps the Milley paper ``Stationarity is Dead''}

This is a good point, we have added a citation by one of the `Stationarity is Dead' authors, Julio Betancourt, that includes the water resource management discussion of that paper, but also focuses across more systems on stationarity and climate.\\

\emph{Line 134: Here, and in a few other places, it is mentioned that ``invasive species are more plastic'' than native species; though a citation to Davidson 2011 is provided early on, there is little other support for this major theme of this paper. In other words, if phenotypic plasticity under variable or changing environments is indeed a likely mechanism for invasion, then additional support for this potential mechanism seems appropriate.}

Agreed, we have added several citations throughout the paper to \citet{funk2008}, \citet{Hierro:2009up} and \citet{wainwright2013}.\\

\emph{The paragraph starting on Line 128 seems a bit internally redundant.}

Agreed, though our phrasing is to make sure our assumptions and predictions are clear and flow logically from one to the other. We have shortened and clarified the paragraph as much as we could while still meeting these goals.\\

\emph{Sentence starting on Line 234 is somewhat unclear and awkward.}

Agreed, we have adjusted this sentence; it now reads as:
\begin{quote} 
 In many systems where both temperature and precipitation shape growing season dynamics the late season can have high drought stress; we do not expect significant invasion during this window because non-native species may not be as adapted to high drought stress compared to the native species in these systems \citep{alpert2000}.
\end{quote}

\emph{Line 249: ``mesic temperate biome'' could easily include grasslands, which in a later phrase in this sentence are implied as belong to a different biome. I assume mesic temperate biome, in this case, is meant as supporting deciduous trees....though this may simply be a matter of semantics, I might suggest modification and clarification of terminology to minimize confusion.}

In our case these mesic temperate biomes include habitats with a mix of forests and meadows so switching to anything specific about trees is not possible. We have adjusted the text for clarity, however, and believe this should remove confusion:
\begin{quote}
Relatedly, multiple studies using various methods now show that in many mesic temperate biomes, invaders are highly sensitive to temperature---tending to advance their phenology significantly more than native species \citep{Willis:2010al,wolkovichAmBot2013}, though, again, this does not appear to be the case in the temperate grasslands \citep{wolkovichAmBot2013}.
\end{quote}

\emph{Table 1 is very good.\\
Figure 1 is much easier to read, and Figures 2 and 3 are straightforward and interesting and easy to read.}

Thank you!\\
\clearpage
\noindent {\bf Referee 2}\\
\\
\emph{I reviewed the initial submission of this mini-review of plant invasions, phenology, and climate change, and expressed skepticism about its novelty and utility in guiding future research. In my review of the revised ms I tried to give the authors the benefit of the doubt, given the enthusiasm of the first referee. In that light I hope my specific comments below are useful. Otherwise I still don't see how this paper contributes to our understanding of how invasions and phenology interact. It seems obvious that warming shoulder seasons may create opportunities for invasive species--simply stating that invaders may be more ``plastic'' and thus better able to take advantage of such opportunities is neither novel nor fully explanatory, given that one then has to show why invaders should be more plastic.}

\emph{And without being too critical, I still don't understand why pointing out that phenology is part of a syndrome of traits gets us closer to uncovering the mechanisms alluded to in the abstract. To the contrary, I don't think the authors have given due consideration to the idea that changing phenology is a `passenger' rather than `driver' of invasions, and different phenology per se may be only one component of larger shifts in the ecology of invaded ecosystems.}

We thank the reviewer for their candor, and the critical thought put into this review. To clarify, we believe recognizing phenology as one of many correlated traits will be critical to understanding its role in invasions. Elucidating the passenger versus driver model \citep{MacDougall:2005oe}, for example, is inherently about recognizing correlated shifts and working to understand which is the actual driver. In this case we do argue that increasing evidence of the differing phenologies of non-native (versus native) species suggests a role for phenology---or a suite of correlated traits---to be implicated in species' and communities invasions. Our stressing of identifying the mechanism and of correlated traits seems very in line to us with the underlying spirit of \citet{MacDougall:2005oe}.\\

\emph{50 Al-Mufti}

Thanks for catching this; we have fixed the citation.\\

\emph{65 biotic -> abiotic?}

This is a great catch. Thank you, we have changed it.\\

\emph{68 here and elsewhere I think the authors need to consider the possibility that certain phenological strategies may not be adaptive with more depth (beyond simply citing Lechowicz 1984 in line 104)}

This has been discussed in great depth by both \citet{Lechowicz:1984cr} and \citet{Ollerton:1992kf}, both of which we cite. Given space limitations we believe referring readers to these excellent papers is a better option than a far too cursory discussion of the topic.\\

\emph{74 What is a trait network?}

We have changed this section heading to ``Trait correlations with phenology.''\\

\emph{74ff I'm not sure I understand the need for this section, which says that phenology is linked to other traits, and that this is well known.}

We agree this seems obvious, but wrote this section because we were aware of no other publication that had synthesized these trait correlations.  If phenology is indeed correlated with other traits related to invasions or invader impact, this is important information.\\

\emph{128ff These seem to be a mix of both hypotheses and assumptions. Eg, if you assume invaders are more plastic, then all sorts of advantages emerge in relation to phenology and probably all other environmental or biotic challenges. And is competitive intensity here just the inverse of high rates of disturbance or the presence of stress?}

Given these comments and those of referee 1 we have rewritten this section for clarity. We clearly define stress and its relation to disturbance in lines 68-70 and do not consider these as simple a reverse of competition.\\

\emph{141 High stress I understand: growth is constrained by cold temperatures. High disturbance I don't understand, because plants have evolved in response to historical disturbances in the form of cold-induced damage, so they wait to develop. Is there evidence to suggest that plants in seasonal environments are more likely to lose biomass in the spring than the summer?}

Stress and disturbance have been variously defined throughout their history of use in ecology and may cause part of the confusion here. We have attempted to alleviate some of this confusion by clearly stating our definitions early in the manuscript (lines 68-70): ``Stress, as generally defined, does not lead to major tissue loss while disturbance does---thus the best definition of stress versus disturbance is often species and location specific \citep{crainebook}.''

Further, recent \citep{Inouye:2008gj,Augspurger:2009gj} work shows plants often leaf out and suffer cold-induced damage. So, while we agree with the reviewer that ideally plants would have evolved to always avoid such damage, it is clearly not \emph{always} the case. These studies also add to much evidence \citep[e.g.,][]{Korner:2003} that tissue loss to cold temperatures is higher in seasonal environments in the spring than in the summer. \\

\emph{150 ``few species leaf and flower early''--isn't this by definition, or we wouldn't call it `early'?}

We believe the definition of `early' is temporal and not directly related to the percentage of species in leaf or flower. While it is true that there will always be tails to the distribution there is variation across systems in what percentage of the species within a community leaf/flower within the first several weeks of the growing season, for example. We believe it would still be called `early' in the season even if half the species started flowering right at the beginning.   \\

\emph{160 Couldn't enemy release be used for advantages during any season?}

It could be, but many studies have found susceptibility to herbivory is especially high in the early season, thus we expect this to be the most critical time. We more clearly state this now and have added two references:
\begin{quote}
Thus, if the early season is a critical period of susceptibility to herbivory \citep{Feeny1970,Barbehenn2013} this may also be the critical time for invaders to benefit from herbivore release.
\end{quote}
% From Judy Myers: The classic paper on food quality and spring is Feeney in Ecology in 1970 - Seasonal changes in oak leaves. Mattson  had a review on leaf nitrogen in ARES in 1980. A recent study http://link.springer.com/article/10.1007/s00049-012-0119-5#page-1 will allow you to track back the literature. 

\emph{191 invasions via phenology is unclear. Meaning, invasions due to enhanced resource uptake compared to natives between cold periods?}

We have clarified this sentence (lines 194-196). It now reads:
\begin{quote}
This high abundance of species in flower, however, means it is also the period of high competition for resources (Fig. 2) and thus we generally predict few invasions driven by phenology mid-season (Fig. 3). 
\end{quote}

\emph{197 This seems to imply that drought stress is a bigger environmental challenge than cold stress-intended?}

Not necessarily. Variation in stress and how much of a challenge plants perceive it as, is certainly a complicated issue. As we mention in the manuscript, ``the best definition of stress versus disturbance is often species and location specific \citep{crainebook}.'' Thus our intention here is merely to point out that in such systems plants may experience very high stress via drought. \\

\emph{202ff It's not clear why invaders should not obey the same tradeoffs.}

This is a good point, we suggest that invaders may not obey the same trade-offs because evolution in different habitats could lead to different trade-offs. We now mention this early on in the paper (lines 48-51):

\begin{quote}
Here we use the `non-native' to refer to any species established outside of its home range; such a distinction between native and non-native is important because non-native species have evolved in a different community than the one into which that have been introduced. Thus, they may exhibit differing strategies and trade-offs than native species.  
\end{quote}

\emph{226 I think the reality is much richer. The authors may want to reflect on the ubiquity of invaders with late season niches that predate modern climate shifts. There are lots of reasons niches can be `open' without climate change-see Mack's 2003 Int J Plant Sci paper, for example.}

We appreciate the referee's point, however, we stand by this prediction.\\

\emph{230 I really think the invader plasticity argument here is a cop out-one can take any process of community assembly and then state invaders have the edge because they are more plastic. If the authors want to continue to emphasize this argument, it would be fruitful to argue what limits plasticity in the natives. Why shouldn't every species be phenotypically as plastic as possible? What's the fitness cost to plasticity in flowering or leaf out times?}

We greatly agree with the reviewer that it will ultimately be critical to understand both the costs and benefits of phenological plasticity in a non-stationary (i.e. changing) climate. We have offered one possibility and have adjusted a key sentence related to this (lines 119-121): ``Non-stationarity could change the optimal phenological strategy---both in absolute timing, and in flexibility in this timing---leaving native species less well-adapted to their current environment and providing an opportunity for invasions.'' Fundamental questions of what drives variation across phenological events, species and biomes in the plasticity of phenology are important, but outside the scope of this manuscript.\\

\emph{240 Use of `results' implies there was a question and a test. What question are these results to answer?}

This is a good point. We meant it to contrast with the previous section about predictions. For the contrast to be clear however, we should have used `predictions' before each subsection in the preceding section, which we now do.\\

\emph{258 The mid season section seems overly focused on flowering and Konza Prairie. Is the aim of this to comment to address the role of mid-season drought on invasions in general? If so there is a world beyond prairie grasslands, such as pines in the S Hemiphere, C4 African grasses in seasonal savannas worldwide, etc, and esp their interaction with fire regimes. In other words, this section seems of very narrow scope for such a large question.}

Good point, we now adjusted the first sentence (line 267) of this section to mention `North American' and have added a sentence to the paragraph:
\begin{quote}
These patterns are based on findings in North American prairie systems, however; far more work is needed additionally to understand if this period is occupied by invaders in other systems or is possibly too stressful. 
\end{quote}
We did have a much longer section related to this, which we hoped to include (shown below), but, again, due to our word limit we had to significantly shorten this into the one sentence mentioned above.
\begin{quote}
These patterns are based largely on findings in North American prairie systems and further studies in additional ecosystems with mid-season droughts may reveal other patterns. The phenology of southern hemisphere savannas, for example, have a long history of phenological study as one of the most strongly seasonal biomes \citep[reviewed in][]{Sarmiento1983}, and one where fire often plays a key role in determining species phenology. While invasions have strongly altered fire regimes in many savannah systems \citep{DAntonio:1992pj}, we know less about how invasions have altered the phenology of savannah communities.  In South Africa most native grasses have C4-type photosynthesis while most non-native grasses have C3-type photosynthesis \citep{milton2004}.  This suggests that non-native grasses in South Africa could be active earlier in the season than the native community because in North American prairie systems C3 species flower earlier on average than C4 species \citep{Craine:2012kl}, but a formal evaluation of the relationship between photosynthetic traits and phenology across wider geographic areas has not been done.
\end{quote}

\emph{279 Again, explain how these invasions are linked to climate change if the spread of the species in question predates modern warming.}

We appreciate the referee's point, however, most of the species mentioned in this section have spread and/or become invasive on timelines consistent with climate change playing a role in these invasions and/or their major impacts. We are now more clear about our definitions, however (see response above). Additionally, we note that climate change has been occurring for $>100$ years in many systems with urbanization, thus the timescale of when shifting climate could play a role in invasions may be longer than traditionally considered. \\

\emph{377 There are two studies cited... is this then a ``commonality across climatically diverse systems''?}

We appreciate this concern, however, in this case one of the studies covers five sites (of disparate climate types) and the other one studies one additional system that is distinct from the five of the other study. Thus, these two publications represent studies of 6 climatically diverse ecosystems.

\newpage
\bibliography{/Users/Lizzie/Documents/EndnoteRelated/Bibtex/LizzieMainMinimal}

\end{letter}
\end{document}
