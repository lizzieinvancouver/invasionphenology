\documentclass[11pt,a4paper,oneside]{article}
\renewcommand{\baselinestretch}{1.8}
\usepackage{sectsty,setspace,natbib,wasysym} 
\usepackage[top=1.00in, bottom=1.0in, left=1in, right=1in]{geometry} 
\usepackage{graphicx}
\usepackage{latexsym,amssymb,epsf} 
\usepackage{epstopdf}
\usepackage{exceltex}
\usepackage{lineno}
\usepackage{gensymb}

% i am checking a move

\begin{document}
\bibliographystyle{/Users/Lizzie/Documents/EndnoteRelated/Bibtex/styles/amnat}
\noindent Open Access: Invited review\\
\noindent Title: \emph{Phenological niches and the future of invaded ecosystems with climate change}\\
\noindent Running head: \emph{Phenological niches and plant invasions}\\
\\
\noindent Elizabeth M. Wolkovich,$^{1}$* \& Elsa E. Cleland$^{2}$ \\
\noindent \emph{$^{1}$Biodiversity Research Centre, University of British Columbia, Vancouver, BC, Canada; $^{2}$Division of Biological Sciences, University of
  California -- San Diego, La Jolla, California, United States of America.}\\
\noindent *corresponding author: lizzie@biodiversity.ubc.ca\\
\\
\noindent Word counts: 206 in abstract; 5,990 in main text\\
\\
%\noindent Last tinkering: \today 
\begin{abstract} A growing body of literature in invasion biology has documented phenological differences between native and exotic species. Multiple studies have found exotic species that tend to be active distinctly early or late in the growing season, advance more with warming than native species, or have shifted earlier with climate change compared to native species. To date, however, work has focused mainly on documenting these phenological differences, with fewer studies framing and testing models where distinct phenologies would drive invader success. We suggest that progress towards such models could be rapid, but would benefit from more efforts to integrate studies into the frameworks provided by basic life history and plant strategy theories. Here we lay out how phenology fits as an important trait affecting how plants balance aquisition, allocation and loss, especially the role of phenology in avoiding and moderating how plants experience disturbance and stress---via climate, herbivory and competition---throughout the growing season. Work on plant invasions and phenology within this framework would provide a more rigorous test of what drives invader success, while at the same time testing basic plant ecology theory. Additionally, extensions could provide the basis to begin to model how ecosystems themselves may shift in the future with continued climate change. 
\end{abstract}


\noindent \emph{Keywords:} exotic, alien or non-native species, phenology, plasticity, climate change, invasions, temperate systems

\newpage
\linenumbers
\noindent Understanding the forces that allow species to invade established communities is a central goal of ecology \citep{Elton:1958bk}, and critical to mitigating impacts of invasive species \citep{Levine:2003cn}. Theories regarding mechanisms of species coexistence and community assembly have helped develop frameworks for predicting when and where invasions are likely to occur \citep[e.g.,][]{Shea:2002pv}. In species-rich communities numerous factors are likely to influence invasion success, including competition with established species for limiting resources \citep{Macarthur:1970kp,tilman1982,tilman1988}, interactions with higher trophic levels \citep{Keane:2002uz,Colautti2004}, and processes associated with environmental variability \citep{chesson1981,chesson1986}. Invasion biology has thus been challenged to to predict which factors may most often be important to invasion success and how they vary across space and time. Recently, theories regarding fluctuating resources \citep{Davis:2000tg} and ``windows of invasion opportunity'' \citep{drake2006,caplat2010} suggest that phenology (timing of life history events) can play a critical role in invasion success \citep{wolkovich:2010fee}. \\
\\
% In the field of invasion biology the long-standing quest for invasibility traits---aspects of species that contribute to their establishment in habitats outside of their native range---has found recent success in the arena of plant phenology. 
A growing body of research focused on plants has found differences between non-native species relative to their invaded communities with respect to phenology, especially leafing and flowering times. Several studies have found that especially early \citep{mcewan2009,Wilsey:2011cr,throop2012,Wainwright:2012tw} or late \citep{Godoy:2009dz,Fridley:2012fj,paquette2012,Pearson:2012uq} phenologies may aid the success of invasive species; while more recent work suggests that exotic species may be the major drivers of longer growing seasons in North America \citep{wolkovichAmBot2013}. These studies generally work implicitly or explicitly from the concept of a temporal niche (Fig. 1): that time is an axis by which species may partition resource use \citep{gotelli1996}. Extensions of this basic niche theory have suggested how such distinct phenologies may result in a competitive advantage for exotic species, especially in areas with shifting growing seasons due to to climate change \citep{wolkovich:2010fee}.\\ 
\\
To date, however, work has focused mainly on documenting phenological differences between native and exotic species within invaded communities, with less effort applied to articulately framing and testing models where especially distinct (e.g., earlier or later than most of the native community) phenologies are a fulcrum for exotic species' success. This is a critical gap since connections to such theory would at once provide a framework to test theory from basic plant ecology and invasion biology, and link to a predictive framework of how phenology, alongside a suite of other plant traits, may shift with climate change. Such extensions would then provide the basis to begin to model how ecosystems themselves mays shift in the future with continued climate change. We suggest that progress at the intersection of invasion biology and plant phenology could be rapid, but would benefit from far more efforts to integrate studies into the frameworks provided by basic life history and plant strategy theories.\\
\\
Towards this goal, here we lay out how phenology fits within major plant strategies, especially the role of phenology in avoiding or mitigating various flavors of disturbance and stress, and in plant competition. While reviewing this we highlight how modern perspectives from evolutionary biology and correlated trait networks may provide novel perspectives for re-examining basic plant ecology theory through the lens of plant phenology, invasions and climate change. \\
\\
\noindent {\bf Phenology within plant life history theory}
\begin{quote} 
\emph{Following the lead of Darwin (1859), many
biologists have focused on competition, predation, disturbance,
and disease as the principal architects of life-history
evolution . . . . Others
tend to view phenology and allocation traits as adaptations
to abiotic environmental condition.} \\
-- \cite{stanton2000}
\end{quote}
Phenology operates within a special realm of plant life history theory---as it affects plant interactions with both biotic constraints (e.g, competition, herbivory, pollination) and abiotic constraints (e.g., frosts, drought). Extensive work over the past several decades has examined phenology from the viewpoints of competition \citep[e.g.,][]{Rathcke:1988yc,VANSCHAIK:1993uq} and mutualisms \citep[e.g.,][]{Brody:1997ro}. Similarly, traditional theories in invasion biology have focused heavily on biotic interactions---in particular competition between functionally similar species \citep[limiting similarity and competitive exclusion,][]{macarthur1967,abrams1983}, or interactions with consumers and pathogens \citep[enemy release,][]{Keane:2002uz,Liu:2006kj}. With the rise of studies documenting plant responses to climate change, the balance has shifted recently towards a more abiotic focus within both the disciplines of phenology \citep[e.g.,][]{Inouye:2008gj,Miller-Rushing:2008zv}, and invasion biology \citep{sorte2013}.\\
\\
% Elsa made changes to the below to kill messiness. 
Plant strategy theory, however, suggests that plants do not generally view the world through dichotomous lenses that can focus only on abiotic or biotic factors and instead through a single model of how these forces affect acquisition, allocation and loss. This model has often then been extrapolated into a focus on how well plants handle competition, stress and disturbance \citep{Grime:1977sw,crainebook}. These terms are messier and more complex than they may appear on first glance: in particular, the lines between stress and disturbance are highly nuanced. Stress, as generally defined, does not lead to major tissue loss while disturbance does---thus the best definition of stress versus disturbance is often species and location specific \citep{crainebook}. Such complexity, however, is necessary when discussing drivers of plant traits, including phenology, as diverse species in varied environments must balance the need to minimize stress or avoid disturbance while successfully competing for light, soil nutrients and water.\\ %We generally follow these terms as we believe they provide an excellent base arena to discuss the role of phenology in plant strategies. 
% Here we follow these definitions, but focus much of our time on stress---which we consider a critical precursor to disturbance for many species (FIX).
\\ 
Considering plant phenology through the lens of competition, stress and disturbance the lines between abiotic and biotic factors quickly blur. For example, both frost and herbivores may act as a disturbance around which plants must balance their leafout timing. The major difference then between abiotic and biotic factors comes via the feedbacks possible with biotic factors (e.g., herbivores may adjust, within their own set of climatic limits, to match earlier leafout). As abiotic factors have little possible feedbacks then, in most systems where abiotic limits have had relatively stationary properties across years---in timing and variability, especially, we expect plants to have adjusted strategies to these system properties.\\
\\
For many ecosystems climate has shifted directionally over recent decades (e.g., increasing temperatures in many systems), meaning that most ecosystems can no longer be safely assumed to have consistent climate means with some stochastic variation around those means. This non-stationarity could fundamentally change the optimal phenological strategy and possibly explain why phenology may be tied to exotic species' success. Indeed, meta-analyses suggest that exotic species generally possess greater trait plasticity than native species \citep{Davidson2011}, such that exotic species may benefit more than native species from global changes that increase resource supply or result in more favorable growth conditions \citep{Richards:2006si}.\\
% This non-stationarity could fundamentally change the optimal phenological strategy and possibly explain why phenology may be tied to exotic species’ success.
\\
Understanding the role of phenology in plant invasions thus requires an explicit model of how competition, stress and disturbance shift across a growing season and with climate change. Such a model should make basic, but testable predictions about when (within a growing season and over longer timescales), in which systems, and how phenology may contribute to invader success. We lay out general predictions below, but focus especially on two types of systems---northern, temperate mesic systems and temperate grasslands---which have received a large amount of study on phenology and invasions to date and provide useful contrasting examples of how competition, stress and disturbance may apply to predicting how phenology may relate to community assembly and invasions. We divide our discussion temporally: focusing first the early portion of the growing season, then middle, and late. We focus on systems with generally one pronounced growing season each year, but provide extensions to systems with bimodal or less clearly defined growing seasons.\\
\\
\noindent \emph{Early-season species}\\
In systems where growing season length is fundamentally structured by low temperatures (e.g., temperate, boreal and arctic biomes), the early growing season should represent a period of high stress and possible disturbance, and---relatedly---low competition. In temperate systems a rise in temperatures signaling the onset of spring is correlated strongly with increased variability in temperatures (see Fig. 2), meaning species with early-season phenologies must have mechanisms to cope with such high variability, or risk losing tissue to frost damage or other extreme temperature swings present in the spring \citep{Linkosalo2000,Augspurger:2009gj}. Further, plants often suffer the greatest herbivore damage early in the growing season \citep{Lechowicz:1984cr}; this may be partly due to a greater apparency to herbivores \citep{Brody:1997ro}. Further, much evidence shows that young leaves are preferred by herbivores, and suggests selection for herbivores to appear during budbreak in temperate ecosystems \citep{vanasch2007}. Early-season phenologies, however, also benefit from reduced competition. Across habitats, few species leaf and flower early (see Fig. 2) yielding lower competition for soil and light resources and for pollinators \citep{Mosquin1971}. Additionally, competition with microbes for soil nutrients may be especially low as in many systems the soil microbial community turns over with the warm spring weather producing a flush of soil nutrients \citep{Zak:1990ar}. Work to date has not clearly documented whether pollinators are as common or effective early, as compared with later, in the growing season \citep{Mosquin1971,filella2013}, but \citet{Mosquin1971} suggests that the earliest-flowering species face low competition for pollinator resources.\\
\\
Early phenologies of exotic species may thus succeed through two major mechanisms independent of climate change. First, invasive species that are active early in the growing season may be particularly successful because they have the opportunity to preempt space and soil resources, grow quickly and shade later-active species \citep{Weiner1990,Wilsey:2011cr,wolkovich:2010fee}. This type of asymmetric competition could create ``seasonal priority effects,'' one mechanism by which exotic species could establish and rise to dominance in a new community \citep{Wainwright:2012tw}. Second, invaders may succeed via an early-season enemy release mechanism. In invasion biology the enemy release hypothesis suggests that exotic species could be less susceptible to herbivory than native species due to a lack of specialist herbivores \citep{Keane:2002uz,Liu:2006kj}. Thus, if the early season is a critical period for herbivory this may then also be the critical window of time then when invaders benefit from herbivory release. This suggests a mechanism by which exotic species could break the risk-benefit trade-off of early phenology experienced by native species, thereby giving them a special advantage from early-season phenology.\\
\\
Climate change may additionally promote early-season phenologies and provide a mechanism whereby exotic species with early-season phenologies are especially successful. Recent increases in spring temperatures in the temperate biome \citep[at least partially associated with increased in greenhouse gases, see][]{Trenberth:2007hk} are the most studied climate change effects on plant phenology \citep{Pau:2011wd}, yielding extensive estimates of how much temperatures have shifted, and how species have responded \citep{Parmesan:2003dm,Root:2003kl,Menzel:2006sq}. Most studies find that spring temperatures have increased as much or more than temperatures in other other seasons \citep{cohen2012}---meaning this is a season of especially high non-stationarity in climate (Fig. 1). Such high variability may make it an optimum period for invasion of other species---since high non-stationarity should mean native species are being pushed far away from the previous long-term climate means to which their phenology should be adapted. Work to date suggests this may be the case in mesic temperate biomes, where invaders are generally early-season species, though this has not been seen in temperate grasslands \citep{wolkovichAmBot2013}. Early-season species, whether native or exotic, also tend to be the most sensitive to temperature \citep{Cook:2012,Wolkovich:2012n}. Relatedly, multiple studies using various methods now show that in mesic temperate biomes, invaders are highly sensitive to temperature---tending to advance their phenology significantly more than native species \citep{Willis:2010al,wolkovichAmBot2013}, though, again, this does not appear to be the case in the temperate grasslands \citep{wolkovichAmBot2013}. An important need in understanding the role of phenology in plant invasions for temperate biomes will, thus, be teasing apart early-season phenology from its associated high tracking of climate. That is, to scale up to predicting invasions it will be critical to understand how much invasion success occurs via this flexibility in phenology (i.e., the trait of the invader leads to success) versus via open niche opportunities present in the early-season due to non-stationarity with climate change (i.e., the system is open to invasion by early-season species), or a mix of the two properties.\\
\\
A coherent picture of how species respond to shifts in spring temperatures, however, will need to include a focus not just on mean or aggregated temperatures, but also on extreme events. Though several studies now show that species that tend to advance with warming also tend to increase in abundance or performance \citep{Cleland:2012vn}, including invasive species \citep{Willis:2010al,chuine2013}, other work shows that the early-season (native) species most sensitive  to climate are those that suffer the greatest performance losses with climate change \citep{Inouye:2008gj}. In this latter case---a subalpine meadow system where phenology is dominantly controlled by snowmelt---a shift in earlier springs that did not coincide with a shift in frost dates produced the performance declines \citep{Inouye:2008gj}. Accurate predictions of which systems may have viable early-season open niches for invasions will require examining how much temperatures have shifted on average while also considering important spring climate events related to plant stress and disturbance, such as frost. To date, increases in frost risk with spring warming have been documented in North America \citep{Inouye:2008gj,Augspurger2013}, but not in Europe \citep{Menzel2003a,Scheifinger2003} or China \citep{Dai2013}, where shifts in freeze and frost days have occurred more in step with the climate shifts driving earlier spring onset \citep{Dai2013}. Far more work in this area is needed as these studies still cover only a relatively small portion of the globe, and, further, there is little known about how early-season exotic versus native species cope with frost and frost risks.\\
\\
\noindent \emph{Mid-season species}\\
In most systems the mid-season represents the period when the greatest number of species flower \citep[e.g.,][]{Fitter:2002sm,Morales:2005ex,Miller-Rushing:2008zv,Aldridge:2011}, and work to date using community datasets suggests that this is true for both native and exotic species \citep{wolkovichAmBot2013}. This is perhaps not surprising as in most systems the mid-season represents a period of relatively low stress due to the physical environment and high abundance of pollinators. On the other hand, this high abundance of species in flower means it is also the period of high stress due to competition for resources (Fig. 2). Basic invasion theory, built on limiting similarity theory, suggests that species should invade during times when most other species are inactive \citep[vacant phenological niche, see][]{wolkovich:2010fee} and to date few studies have linked phenology to invasion in the mid-season.\\
\\
The mid-season is also the major period where temperate mesic and grassland systems tend to bifurcate most notably in their climates (Fig. 2). In mesic systems temperatures may peak mid-season but there is generally no major shift in the precipitation regime. In contrast temperate grasslands are generally characterized by mid-season droughts when the highest temperatures often coincide with a decline in precipitation, producing large declines in soil moisture \citep[e.g.,][]{Craine:2012kl}.\\
\\
Several studies have documented declines in native species alongside shifts with climate change in mid-season drought periods---generally showing that the drought period may be extending or becoming more pronounced \citep{Aldridge:2011} and multiple authors have postulated that this period will result in a vacant niche for invaders \citep{Sherry:2007fq,Aldridge:2011}. To date, no studies (of which we are aware) have shown invaders occupying these mid-season drought periods, while in contrast two studies of phenology at Konza Prairie LTER have found a decline in exotic flowering coinciding with the mid-season drought \citep{Craine:2012kl,wolkovichAmBot2013}. Far more work here is needed to understand if this period is occupied by invaders in other systems or is possibly too stressful. This represents an area where predictions are highly difficult for several reasons: (1) understanding and modeling how species respond to moisture has proven far more difficult than modeling temperature responses \citep{Crimmins:2011lq,wolkovichAmBot2013}, (2) work to date suggests phenological responses to drought are highly variable between different species \citep{Jentsch:2009ff,Prieto:2009fu}, and (3) projections of how precipitation and droughts will shift in the future are some of the most uncertain of all climate change forecasts \citep{knutti2013}.\\
\\
\noindent \emph{Late-season species}\\
Climatically, in many temperate systems the end of the growing seasons mirrors the beginning (Fig. 2); plant phenology is, however, highly different. While in both temperate mesic and grassland systems plant leafing and flowering (Fig. 2) generally tracks closely the rise in spring temperatures around 5$\,^{\circ}\mathrm{C}$, community flowering curves generally end at least one month before temperatures return to 5$\,^{\circ}\mathrm{C}$. The reasons for this may be simply related to physiological constraints: species must flower well before the end of the season in order to have enough time to fruit and set seed. A balance between risk and investment may also be important: as mean temperatures drop, temperature variability climbs, just as low spring temperatures are also correlated with higher temperature variability. In contrast to the spring, however, later in the season almost all species have made a heavy investment in growth and reproductive tissues and loss to frost may pose an even greater threat than frosts in the spring, thus plants may be especially conservative. \\
\\
Recent studies of plant invasions, however, suggest exotic species may not play by the same risk-investment rules as native species in the fall. A pervasive exotic understory species in the Ohio River Valley, \emph{Lonicera maacki}, stays green later than any native understory species, allowing mapping of \emph{L. maacki} by satellite in the fall \citep{becker2013}. A study of several dozen North American exotic species from the understory confirms that these species consistently senesce in the fall later than native species \citep{Fridley:2012fj}. \citet{Fridley:2012fj} also found that many of these species remain green until the first major frost and thus lose their leaves to frost versus plant-induced senescence. This should be a major cost to the plant---as most species resorb nutrients well before the first frost \citep{Lambers:2008jb}. Further work showed that the longer leaf life-span of the exotic species enabled a greater time-integrated nutrient-use efficiency \citep{Heberling2013}, and, coupled with high rates of nutrient uptake the following spring, provides a mechanism by which the unique phenology of exotic species, compared to the native community, could promote invasion. \\
% and understanding how these species may actualy benefit from this strategy could help improve understanding of the mechanism by which a unique phenology, compared to the native community, may promote their invasion.
\\
If these exotic species in eastern North America do gain a large benefit from occupying an open late-season temporal niche, even while losing tissue to frost, then climate change could further increase their success. In many habitats, fall temperatures are rising more quickly than even spring temperatures \citep{cohen2012}, and are expected to continue to rise, further extending the open niche space at the end of the season, which already appears temporally much greater than the early season. Thus it appears to be a period of very low plant competition and may also be when microbial communities again turnover \citep{Bardgett:2005ls}---suggesting that species which can remain active until the end of the season may have access to a large available resource pool. \\
\\
Across systems, however, autumnal shifts in climate and phenology are still relatively unstudied compared to spring. More work is needed to understand where, and by how much, mean fall temperatures are shifting in comparison to other seasonal temperature shifts, and how species are adjusting their late-season phenological events. In particular, there is very little work from the late-season in grasslands, where the combination of shifts in mid-season droughts alongside shifted fall temperatures may create novel climates and, possibly, novel opportunities for invasion. Alternatively, systems with mid-season droughts may have reduced opportunities for invasion if such droughts---over the long-term---have favored more variable phenological strategies \citep[e.g., species flower very late to avoid stress of mid-season drought, see][]{Craine:2012eco} compared to simple temperature-controlled systems, which have been noted to have far greater late season empty phenological space compared to grasslands \citep{Craine:2012eco}. Finally, we are aware of no work examining how frost dates and other late-season climatic events have shifted with climate change, but understanding the prevalence and magnitude of such shifts should be critical to understanding and predicting how late season phenology may affect invasions. \\
\\
\noindent \emph{Beyond the temperate biome}\\
We have focused above on two of the more highly studied ecosystems---temperate mesic and grassland systems---for phenology and plant invasions, yet the opportunity for phenology to impact plant invasions clearly extends to other systems. In particular systems with variability in when the start of season occurs and in the related magnitude of the climate event that starts the season (e.g., mm of first major rainfall in many semi-arid systems) may be particularly prone to early season invasions, especially if climate change has shifted climate dynamics \citep[for a longer discussion see][]{wolkovich:2010fee}. Phenological enemy release should also be considered as a possible mechanism for invader success in any system with variation in herbivory pressure between or during seasons. For example, an early-season enemy release mechanism may also extend to tropical systems with a pronounced wet-season, where both plant and herbivore phenology may be cued by rainfall. Research using 32 species of tree saplings in Panama showed variation across the season in herbivore pressure (lower during the dry season) and reduced herbivory on species that flushed synchronously during the wet season \citep{aide1993}.  Although untested, these results suggest that tropical exotic species could benefit from phenological enemy-release.\\
\\
To date, work in semi-arid systems suggests invaders may occupy especially early-season niches, but such studies are still relatively rare compared to work in temperate systems. Additionally, there is a paucity of work on phenology and invasions in ecosystems where the start of the season is controlled by snowmelt, but such systems are known to have shifted greatly in many areas with climate change \citep[e.g.,][]{pederson2011}. For all such systems a major need for research is more data and study of relevant climate variables especially soil moisture, soil temperature and snow cover---variables that, compared to temperature, are few and far between. These types of data, however, we expect will prove key in improving understanding of phenology in precipitation controlled systems, including alpine, grassland, semi-arid and arid systems.  \\
\\
Studies of precipitation-controlled systems must also deal with large-scale, longer-term cycles in precipitation that often dominate in many such systems (e.g., El Ni\~{n}o in many semi-arid systems in western North America). Such longer-term cycles may be a critical consideration for several reasons. First, native species may be expected to be adapted to them and thus relevant studies of population dynamics will need to work across the relevant climate oscillation timescales. This is important for exotic species as well, as research suggests these oscillations may also dictate population shifts for exotic species \citep{Salo:2005eo}. Second, such oscillations may shift with climate change \citep{ipccPhys2007}, making predictions more difficult. Indeed, precipitation-controlled systems overall are the ones in which predicting how climate will shift in the future has proven most difficult \citep{knutti2013}.\\
\\
\noindent \emph{Shifting playing fields and phenological plasticity}\\
One commonality across climatically diverse systems is evidence linking exotic species with high phenological flexibility. Across both systems where the start of the growing season is determined by temperature \citep{wolkovichAmBot2013} or by precipitation \citep{Wainwright:2012tw} species that track the start of season most closely are highly successful invaders. Understanding the benefits and trade-offs of high phenological tracking of environmental variables would address a fundamental question in invasion biology: if high plasticity yields a competitive advantage, why do species differ in their plasticity? One hypothesis is that species that track climate change well over the timescales for which we have data (or focus our analyses) may suffer major population losses during years of extreme climate. Alternatively, the current non-stationarity of climate may have shifted the playing field; this may mean species that evolved in more variable environments are now often successful invaders. These hypotheses have not been tested to our knowledge. However, adaptations of bet-hedging models, traditionally focused on desert annual plant communities \citep[e.g.,][]{donald2013}, combined with currently-available climate data should allow basic vetting. In particular long-term climate data could provide insight on how high population losses from extreme events must be, and how frequent such events inducing population losses must be to make high phenological tracking a detrimental long-term strategy. Similarly, climate data could form the basis to examine if non-stationarity has significantly shifted the playing fields to make high-phenological tracking the optimal strategy. Such models, however, will only provide basic tests of these hypotheses and require much further data for any useful predictions.\\
\\
Data are needed on exactly how exotic species are able to track the start of seasons more closely---that is, the exact complex of cues they use to to trigger leafing and flowering. While fully identifying phenological cues for any species can be a long process, often involving extensive lab studies as many species appear to have complexes of cues involving temperature, soil moisture and photoperiod \citep{Stinchcombe:2004ec,Wilczek:2010ad}, it is necessary to predict exactly which climate scenarios will trigger growth and thus susceptibility to early-season variability. For example, an exotic species may appear to respond only to temperature based on long-term field observations, but may actually only integrate temperature under certain photoperiods (which could be identified via a suite of lab studies varying both temperature and photoperiod). Native species may similarly respond to both temperature and photoperiod, but the exact mix of of these cues may differ between the native and exotic species, making the exotic species able to track current climate patterns more closely. In addition to improving understanding of how exotic species may track the start of seasons more closely than native species, documenting exact cues may improve predictions from which climates future invaders will come, as trends in photoperiod and temperature cues are often predictably clinal \citep{Howe:2003,Wilczek:2009oa}. For example, returning to the example of photoperiod and temperature cues, exotic species in temperate systems may be expected to generally come from more southerly and more mild climates where photoperiod cues allow species to leaf and flower earlier than more northerly and climatically extreme locations.\\
\\ 
Progress will also come from understanding more quantitatively the risk/benefit of highly plastic strategies for exotic species \citep[e.g., exotic species that senesce later in the fall often lose green leaves to the first frost, which should be a major carbon and nutrient loss, see][]{Fridley:2012fj}. Testing the hypotheses laid out here requires knowing how much tissue gain and loss exotic species suffer across varying climate regimes they experience in their introduced habitats (i.e., across the variability in climate across the growing season and between years), especially in comparison to other species. Identifying what causes shifts in gains and losses---for example losses due to climate stress versus herbivory---will be critical to tying phenology to invasion success. \\
\\
Studies examining the plasticity of phenology in exotic species may also want to consider how much evolutionary change following introduction has contributed to this plasticity \citep[e.g.,][]{sultan2013}, and how quickly species can genotypically adjust more static phenological cues post-invasion. Many of the studies mentioned here examine phenological shifts which are attributable to phenotypic plasticity (e.g., they are of marked perennial individuals or come from woody species known to shift leafing and flowering plastically between years with different climates), but recent studies have documented rapid evolutionary shifts in invaders, especially in phenology \citep[e.g.,][]{Colautti:2011,konarz2012,novy2012}. Understanding how much phenological flexibility is driven by underlying plastic versus genetic shifts is important to projections of which species may become invaders---if much change is genotypic it suggests then that predictions may be more difficult and will require knowing \emph{a priori} how quickly phenology can evolve under new selection regimes. More studies examining invaders in their native and introduced ranges \citep[e.g.,][]{Godoy:2009dz,matesanz2012} would begin to build data on how often phenologies shift with invasions and how similar or distinct invader phenologies are in their native versus introduced ranges \citep[e.g.,][]{wolkovichAmBot2013}.\\
\\
\noindent {\bf Links to traits and ecosystems}\\
\\
Examining plant invasions and phenology from the perspective of how plants balance acquisition, allocation and loss amidst a changing playing field of stress, disturbance and competition has many research benefits. It integrates applied questions within a framework that allows them to advance basic and applied biology at once. It ties more directly to mechanisms, which allows findings to be more easily ported to other relevant systems. We see two benefits, however, as particularly important for research related to phenology: (1)  it forces phenology to be examined from the perspective as one trait within a network of correlated traits and, relatedly, (2) it allows more rapid scaling of how invasive species will affect ecosystems. We discuss the value of each of these in turn.\\
\\
To date, most work on invasions and plant phenology has tended to focus rather fixedly on phenology as a single trait of interest \citep[but see][]{Sun:2011eu,hahn2012}, yet plant strategy theory \citep{crainebook} highlights how phenology is just one in a suite of traits related to strategy. For example, species with low leaf lifespans often also have higher leaf mass areas, suggesting a relatively low investment in leaves and possibly ruderal strategy \citep{Mack:1996ly}. Evidence to date suggests phenology is probably also correlated with other traits, such as plant height, seed mass \citep{Bolmgren:2008vo} and growth rate \citep{Sun:2011eu} and, for temperate trees, traits related to plant hydraulics \citep{Lechowicz:1984cr}. In this latter case, later leaf-out phenologies were associated with wider vessels and/or a greater number of narrow vessels in their wood, compared to native species. Larger vessels can transport water more rapidly but also are more susceptible to breakage during more extreme climate events (e.g., frost or drought). Because many trees must activate their hydraulics systems before they can leaf out each spring, this leads to the hypothesis \citep{Lechowicz:1984cr}
that later phenology may be an enforced requirement for species with larger vessels and highlights how phenology ma be correlated with trait syndromes. For instance, early phenologies may be associated with a lower investment in structural support---they may thus act as somewhat ruderal species with less competitive, but quicker to activate, hydraulic systems \citep{Lechowicz:1984cr} and higher growth rates \citep{Sun:2011eu,hahn2012}, though not all studies have found such correlations \citep[e.g.,][]{Craine:2012kl}.\\
\\
These few noted trait relationships show the potential for how integrating phenology within a more holistic trait framework would enable scaling to ecosystem predictions more easily. If phenology consistently covaries with other traits, either within or across habitats, and phenology is correlated with plant invasions, it would suggest a suite of plant traits that would change in concert with plant invasions and climate change. Recent work shows that temperate exotic species are more phenologically responsive to temperature than native species and that species that advance with warming also tend to increase in abundance and performance \citep{Cleland:2012vn}. Together these findings suggest that future temperate ecosystems may be dominated by more phenologically plastic species. If such flexibility correlates with other traits (e.g., lower leaf lifespan or lower nutrient content in leaves) we would then predict cascading shifts in ecosystem properties such as decomposition and nutrient cycling. Scaling up from phenology and plant invasions to ecosystem-level shifts, however, requires addressing a bevy of untackled questions. For example, it will be critical to know if any noted trait relationships including phenology are similar for native and exotic species, and consistent across clades and plant lifeforms? \\
\\
In a similar vein of the need to consider phenology as just one of many plant traits in an individual's strategy to balance acquisition, allocation and loss, more research is needed that examines links between phenological events. To date much research has focused on flowering, and, secondly leafing---which may be tightly linked in many species (e.g., earlier leafing allows for earlier flowering), with fewer studies still on fall coloring and senescence and very few on fruiting and seeding. This singular event focus is not surprising as often data are only collected on one event. By their very nature, of course, phenological events are each coupled to one another in sequence, which means that examining only one provides only a small sliver of the complete temporal story. \\
\\
Phenological events may also be inherently linked between species, due to their shared evolutionary history---their phylogeny. Research from molecular ecology has consistently shown a strong genetic basis for leafing and flowering times within species \citep{Howe:2003,arabid2011}, thus it may be expected that related species would share similarities in their phenologies, and possibly their phenological responses to climate. Indeed, a growing number of studies have documented significant evolutionary structure in the distribution of flowering times and sensitivities to climate change across species \citep{Willis:2008bf,Davis:2010ls}, including invaders \citep{Willis:2010al,wolkovichAmBot2013}. A recent, more comprehensive analysis across \(>20\) sites, from temperate to tropical Northern Hemisphere zones, shows phylogenetic structure in flowering and leafing for almost all communities studied \citep{phenophylo}. This structure means studies of phenology including multiple species need to consider how much variation in phenology and related phenological traits (such as sensitivity to temperature) is explained by shared evolutionary history versus other factors. This is especially important in studies attempting to link phenology to invasion success---studies finding multiple exotic species with distinct phenologies compared to the native community will need to test how much this finding is driven by phylogeny, as well as the species pool from which exotic species are drawn.\\
%This comes back to trait correlations…. If phenology is linked to other traits, then it’s challenging to figure out mechanistically what’s driving success
 \\
\noindent \emph{Conclusions}\\
With future climate change invasive species are predicted to increase both in abundance and in spatial distribution \citep{ipcc2007summary,bradley2010}. Increasing research in invasion biology has linked phenology to exotic plant species across the globe, while at the same time climate change research has documented dramatic shifts in plant phenology. Both these fields have provided tremendous new data and systems to study further, but both have also tended to focus more on documenting effects, compared to linking research back to basic biological questions. We have outlined here how increasing interest in examining how phenology may affect plant invasions could be re-approached from a more focused framework, one that considers phenology as one component within the constant plant battle to optimally balance acquisition, allocation and loss in a globe where most systems' variable climates are now also highly non-stationary. Much further work is needed to understand how phenology correlates and is constrained by other traits, and whether this varies between different climate regimes, functional groups and clades, and whether exotic species appear to face the same constraints to their phenologies as other species. Fundamentally, we see the strong links recently documented between invaders, climate change and phenology as highlighting how these systems could provide powerful new models to understand phenology as one piece of the plant strategies puzzle. \\
\\
\noindent {\bf Acknowledgments}\\
Comments from J. Craine on earlier drafts improved this manuscript. EMW was supported by the NSERC CREATE training program in biodiversity research.  

\newpage
\bibliography{/Users/Lizzie/Documents/EndnoteRelated/Bibtex/LizzieMainMinimal}


\newpage
\begin{figure}[h!]
\centering
\noindent \includegraphics[width=0.8\textwidth]{/Users/Lizzie/Documents/git/manuscripts/invasionphenology/figures/invasion_niches.png}
\caption{Temporal niche theory suggests that species will partition the growing season to minimize competition for resources (for example, soil nutrients, light, pollinators). Here we show idealized niche diagrams for 10 species (gray and purple distributions) in a hypothesized simple mesic temperate system where temperature limits viable periods for plant growth. Across the growing season variation in stress, disturbance and competition may dictate the optimal phenological strategy: with early-active species experiencing lower competition but also more variable temperatures, in the mid-season community flowering peaks (see Fig. 2) and thus we expect mid-season active species may need competitive strategies to succeed in gaining access to resources. Late season represents a period when many species senesce before variable fall temperatures set in and thus may be a period for species with stress tolerant strategies to access additional resources. With climate change extending viable periods for plant growth (dark blue lines) exotic species (purple, dashed niches) with highly plastic phenologies may have an increased opportunity for invasion at the start and end of the growing seasons in temperate mesic systems.}
\end{figure}

\newpage
\begin{figure}[h!]
\centering
\noindent \includegraphics[width=1\textwidth]{/Users/Lizzie/Documents/git/manuscripts/invasionphenology/figures/invasionphen_commphen_final.png}
\caption{Flowering of species within a community (dashed green line) varies across habitats. Mesic temperate systems such as Chinnor (UK, left) are often defined by temperature (darker blue shading), while other systems such as the tallgrass prairie of Konza (Kansas, USA, right) may have variable drivers across the growing season. In both systems temperature sets the beginning and end of the season and, as such, early-season species show strong sensitivity to temperature \citep{Cook:2012,Craine:2012kl}. In Konza, a consistent mid-season drought, however, coincides with a decrease in the number of species flowering at that time \citep{Craine:2012kl}. We assume, based on many physiological studies, that temperature below \(5^{\circ}\mathrm{C}\) limits development. Standard deviation (SD) of temperature and the coefficient of variation (CV) of precipitation are given as pentad (5-day) averages. Flowering data are species averages from NECTAR \citep{nectar}, climate data for Chinnor taken from GHCN UK000056225 and cover 1853-2001, while climate data for Konza were taken from GHCN USC00144972 and cover 1893-2010.}
\end{figure}

\end{document}




\noindent \emph{How does competition with soil microbes affect phenology?}\\
To date most work on plant phenology has focused on how phenology ties to climate, or plant competition \citep[including for pollinators etc.,][]{Brody:1997ro}, yet it should also fundamentally be tied to soil microbial communities. Indeed, for most early species ties with climate may also, effectively, reflect ties to temporal microbial community dynamics, as soil microbial communities often turnover each spring producing a race for available nutrients between plants and microbes---the so called vernal damn \citep{Zak:1990ar,Lipson:2004es,Bardgett:2005ls}. While work integrating phenology and microbial communities is thin \citep[but see][]{dickens2013,esch2013}, recent research highlights that phenology can be shifted several days by microbial communities \citep{lau2012}. Further, extensive work in invasion biology highlights how soil microbes may play a critical role in promoting plant invasions \citep{Klironomos:2002jg,Wolfe:2005yr} suggesting the intersection of invasions, phenology and microbial effects may provide critical insights to the actual mechanisms by which phenology could affect plant invasions.

THIS SHOULD GO SOMEWHERE generally (since Cook et al. PNAS shows significant vernalization cues throughout the summer): Any such efforts will also probably need to extend back through the growing season -- need to extend  should be noted that for many species, especially perennial species, winter temperatures can also have a dramatic effect on spring phenology


Set up each as: (1) these are the forces and considerations (a-b: try to sort by relative importance), (2) this is what happens with climate change, this means this for invasions (argh, maybe just weave this in or see how it goes?), (3) evidence for this is X, Y, Z. Testable stuff is . .  . .\\
\\
\noindent \emph{Mid-season species}\\
- lower stress and disturbance due to physical environment\\
- but now stress can come from high competition with plants and microbes probably\\

- easier access to pollinators\\
- higher herbivory?\\
- mid-season droughts --> high stress (microbes do what?)\\

- * climate change may open up niches -- some non-stationarity, harder to predict across systems because highly dependent on precip, which is a bitch to predict, and maybe humidity matters or at least drought periods, also hard to predict.\\

- current work shows mid-season is way too stressful for invaders.

\noindent \emph{Late-season species}\\
- high risk of disturbance (rarely stress, things seem to just get killed)\\

- lowering competition\\
- microbial turnover\\

- * I think climatically this has shifted less (ASK Ben) but many plants show conservative strategies so may be big opportunity? \\

- current work shows some evidence of this in North temperate systems

\noindent \emph{Special considerations in systems with bimodal or year-round growing seasons}\\
- Whether there is variability in timing of start of season and magnitude. . .  . understanding soil moisture instead of just precip will be key.\\
- Tropics biotic forces may dominate (seed predation, disease, pollinators, seed dispersers) but small changes in abiotic forces may have large effects. .  . .

- * Some systems critically controlled by large-scale, longer-term weather patterns that may be key to predictions: PDO stuff (Julio's work).

* just means more about climate change, as compared to other stuff about what shapes plant strategies\\


\\
\\
Next sections:\\
Links from plant strategies to traits and ecosystems?\\
evolutionary history (this actually fits with mentioning traits!)\\
See below 'Ways froward' and onward.

\begin{enumerate}
\item Opening
\begin{enumerate}
\item Ditty about how much phenology has come into light with plant invasion research recently
\item Good, but need to focus on questions that integrate into a fundamental framework in order to make more useful, rapid progress
\item Point out Grime's theory here? see Table 2 in Grime 1977
\item Frame paper (`our goal . . . ') as pulling phenology away from climate change descriptions and back to basic life history theory, where it has spent much of its life before the rise of the IPCC in the last 15-20 years (don't say that of course, at least not like that last clause)
\item Big topics will be: reviewing how phenology fits into basic plant competition theory, reviewing recent literature (or not, maybe more reviewing how phenology fits into basic plant competition theory, pointing out along the way how recent literature may fit into this), suggesting progress through:
\begin{enumerate} 
\item competition vs. stress with a . . .
\item traits perspective (efficiences and trade-offs?)
\item microbial-plant perspective 
\item a dash of evolution?
\end{enumerate} 
\end{enumerate}
\includegraphics[]{../../figures/invasion_niches.png}
\item Back to basics: what are we dealing with here? `Principles architects of plant life history theory are: competition, predation, disturbance and disease (following Darwin). Others tend to focus on environmental conditions.' Paraphrased from Stanton et al. 2000. This is crazy contrast though because the environmental conditions (weather etc.) are a `disturbance' of sorts.
\begin{enumerate}
\item plants work on: 
\begin{enumerate}
\item aquisition
\item allocation
\item loss
\item so plants deal with competition and stress (from disturbance, herbivory, environmental variabiity and low productivity)
\end{enumerate}
\item try to view each of these through phenological lens - early vs. competitive strategies for aquisition, allocation and preformed buds or such, and avoiding loss by starting late, but then there's herbivory
\item uh, stress avoidance versus stress tolerators? Are any natives really good stress tolerators? Evidence from Konza and Fargo for no, talk about mid-season troughs and how that reviewer is an idiot
\item Frame sub-headings as (1) early, (2) mid and (3) late season strategies? Go though and review what plants are balancing in different habitats (well, let's just do temperate mesic and grassland) and what current exotic species evidence suggests may be going on. Yeah, yeah, this sounds good -- do this!
\item Early: lots of evidence that early is better (Munguia Rosas et al meta-analysis, Stanton et al. 2000); Grime's ruderal species here?
\item mid: stressful or competitive, but not a hotspot for invaders (if you ask me); Grime's competitive species
\item late: question for self and world: do any species start vegtative growth very late? Basically, when is the latest a species can start up? Seems like most get going by/before mid-summer . . . resources are too scarce after that?
\end{enumerate}
\item Ways forward
\begin{enumerate}
\item open-minded perspective on how plants are succeeding in novel habitats: balancing aquisition, allocation and loss (this could just be a header section before we lay out 3 subsections)
\item focus on plant-microbial competition as well as plant-plant competition
\item Think about correlated traits: Bolgren and Cowan 2008; call for thoughtful trait models, not just throwing everything in and seeing what comes out (this could move up to right under header then, as plant strategy theory should tell you what traits to think about).
\item (3 Aug 2013: This idea eems too half-baked for now, save for future, methinks.) Maybe think about evolution vs. plasticity and habitats under which phenology has evolved (Lechowicz 1984, ``relict requirements'', adaptive signal of phenology in Northern temperate forests may not always be obvious -- `evolutionary disequilibrium' -- but invaders don't seem to have that!)
\item Stationarity angle (goes with previous point somewhat)
\end{enumerate}
\item Closing
\begin{enumerate}
\item invasion biology and climate change research fields have suffered similar fates: prime for testing fundamental questions but burdened by emotive hypotheses. . . climate change ecology has general fallen into similar camps as macro- and microevolutionists: with those focusing on the large scale finding abundance evidence for a strong and deteministic role of climate in shaping species and communities, while local scale focus clearly shows how biotic interactions must critically structure stuff. (See Jablonski 2008.)
\end{enumerate}

\end{enumerate}
Yann's paper: Understory species are more temperature sensitive.\\
\\
Where does this stuff go? From Grime 1977 (note to self, I am not sure I agree with this shit at all, but interesting perspective):\\
- A well-known feature of many xerophytes and, in particular, the succulents is the rarity and erratic nature of flowing [sic].\\
- A characteristics of many shade-tolerant species is the paucity of flowering and seed production under heavily shaded conditions. This phenomenon is particularly obvious in British woodlands, where flowering in many common plants such as \emph{H. helix, Lonicera periclymenum,} and \emph{Rubus fruticosus} agg. is unsually restricted to plants and exposed sunlight at the margins of woods beneath gaps in the tree canopy.\\
\\
{\bf Ask Elsa:} to do a quick sweep of invasions phenology lit and add any important new references.


\newpage
\linenumbers

See cut stuff from temporalchange\_2013.tex. Here's some of it:
\\
Testing hypotheses about trade-offs with long-term climate dynamics or other plant strategies (e.g., early-flowering as a ruderal but not competitively dominant strategy) represents one area where there is immediate need explicit theory\\
\\
We hypothesize, however, that few species will have simple or dominant photoperiod controls for start of season events\\
\\
While these varying approaches may have lead to historical legacy effects\\
\\
Most temperate plant species have survived several expansions and contractions of their ranges over past glaciation cycles (Figure 4) meaning the molecular architecture underlying their phenologies are flexible enough---either through phenotypic plasticity or genotypic shifts, or some combination of the two---to sustain shifting climate and evolving ranges.\\
\\
Would also be great to work in somewhere how multiple phenological cues may only be noticeable at the extremes: for example, photoperiod may control many temperature cues but until warming is extreme (or in a growth chamber you massively muck with the photoperiod cue) you would not be able to note it. Same for some soil moisture issues.\\
\\
cite Martie's paper AmNat 1984 paper: The Quaternary history of the temperate forests has been one of repeated disruptions by glaciations such that present forest communities must be seen as transient assemblages rather than stable sets of coevolved species.\\