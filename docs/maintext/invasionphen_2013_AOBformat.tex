\documentclass[11pt,a4paper,oneside]{article}
\renewcommand{\baselinestretch}{1.8}
\usepackage{sectsty,setspace,natbib,wasysym} 
\usepackage[top=1.00in, bottom=1.0in, left=1in, right=1in]{geometry} 
\usepackage{graphicx}
\usepackage{latexsym,amssymb,epsf} 
\usepackage{epstopdf}
\usepackage{exceltex}
\usepackage{lineno}
\usepackage{gensymb}

\renewcommand{\familydefault}{\sfdefault}

% get refs:
% Naturalization and invasion of alien plants: concepts and definitions (by Richardson, 2000)

\begin{document}
\renewcommand{\refname}{\CHead{Literature Cited}}
\bibliographystyle{/Users/Lizzie/Documents/EndnoteRelated/Bibtex/styles/amnat}
%\noindent Open Access: Invited review\\
\noindent {\Large {\bf Title:} Phenological niches and the future of invaded ecosystems with climate change}\\
% \noindent Running head: \emph{Phenological niches and plant invasions}\\
% \noindent Elizabeth M. Wolkovich,$^{1}$* \& Elsa E. Cleland$^{2}$ \\
% \noindent \emph{$^{1}$Biodiversity Research Centre, University of British Columbia, Vancouver, BC, Canada; $^{2}$Division of Biological Sciences, University of
%  California -- San Diego, La Jolla, California, United States of America.}\\
% \noindent *corresponding author: lizzie@biodiversity.ubc.ca\\
% \noindent Word counts: 267 in abstract (max of 300 I think); 5994 (max of 6000) in main text\\
%\noindent Last tinkering: \today 

\noindent {\bf Abstract:} In recent years, research in invasion biology has focused increasing attention on understanding the role of phenology in shaping plant invasions. Multiple studies have found non-native species that tend to flower distinctly early or late in the growing season, advance more with warming or have shifted earlier with climate change compared to native species. This growing body of literature has focused on patterns of phenological differences, but there is a need now for mechanistic studies of how phenology contributes to invasions. To do this, however, requires understanding how phenology fits within complex functional trait relationships. Towards this goal, we review recent literature linking phenology with other functional traits, and discuss the role of phenology in mediating how plants experience disturbance and stress---via climate, herbivory and competition---across the growing season. Because climate change may alter the timing and severity of stress and disturbance in many systems, it could provide novel opportunities for invasion---depending upon the dominant climate controller of the system, the projected climate change, and the traits of the native and non-native species. Based on our current understanding of plant phenological and growth strategies---especially rapid growing, early-flowering species versus later-flowering species that make slower-return investments in growth---we project optimal periods for invasions across three distinct systems under current climate change scenarios. Research on plant invasions and phenology within this predictive framework would provide a more rigorous test of what drives invader success, while at the same time testing basic plant ecological theory. Additionally, extensions could provide the basis to model how ecosystem processes may shift in the future with continued climate change. \\
%\noindent \emph{Old stuff:} We suggest that progress towards such models could be rapid, but would benefit from more efforts to integrate studies into the frameworks provided by basic life history and plant strategy theories. Here we lay out how phenology fits as an important trait affecting how plants balance aquisition, allocation and loss, especially the role of phenology in avoiding and moderating how plants experience disturbance and stress---via climate, herbivory and competition---throughout the growing season. 

\noindent \emph{Keywords:} non-native, alien or exotic species, phenology, plasticity, climate change, invasions, temperate systems

\newpage
% In the field of invasion biology the long-standing quest for invasibility traits---aspects of species that contribute to their establishment in habitats outside of their native range---has found recent success in the arena of plant phenology. 
\linenumbers
\noindent {\bf Introduction}

\noindent Understanding the forces that allow species to invade established communities is a central goal of ecology \citep{Elton:1958bk}, and critical to mitigating impacts of invasive species \citep{Levine:2003cn}. Mechanistic models of community assembly have helped develop frameworks for predicting when and where invasions are likely to occur \citep[e.g.,][]{Shea:2002pv}, however, numerous factors may influence invasion success, including competition with established species for limiting resources \citep{Macarthur:1970kp,tilman1982,tilman1988}, interactions with higher trophic levels \citep{Keane:2002uz,Colautti2004}, and processes associated with environmental variability \citep{chesson1981,chesson1986}. Further, climate change may facilitate invasion by non-native species \citep{dukes2011}.  While many studies have focused mechanistically on direct positive effects of warming or resource enhancement on invasive species \citep{bradley2010}, there is growing recognition that climate change could facilitate invasions because of the distinct phenology or phenological sensitivity of non-native species \citep{Willis:2010al,wolkovichAmBot2013}.  

Theories regarding fluctuating resources \citep{Davis:2000tg} and ``windows of invasion opportunity'' \citep{drake2006,caplat2010} suggest that seasonal phenology---the timing of life history events---may play a critical role in invasions \citep{wolkovich:2010fee}. Models of invasion success that hinge on phenology build from the concept of a temporal niche (Fig. 1)---that time is a fundamental axis by which species may partition resource use \citep{gotelli1996}, reduce interspecific competition, and thus promote coexistence \citep{Chesson:1997dz}. Extensions of this basic niche theory have suggested how such distinct phenologies may result in a competitive advantage for non-native species \citep{Godoy:2009dz}, especially in areas with shifting growing seasons due to climate change. If native species do not accurately and rapidly track shifting climate, then climate change may produce phenological vacant niches. In brief, such vacant niches may then promote invader success (1) directly (i.e., an invader occupies the open niche space) or in concert with (2) early-season priority effects, via (3) invader plasticity, where non-native species track climate shifts more closely than native species or (4) greater niche breadth \citep[see Fig. 1 and][]{wolkovich:2010fee}. % termchange: invasive to 'non-native species track'

Alongside these theoretical developments, a growing body of research focused on plants has found phenological differences, especially in leafing and flowering times, between non-native and native species. Several studies have found that especially early \citep{mcewan2009,Wilsey:2011cr,throop2012,Wainwright:2012tw} or late \citep{Godoy:2009dz,Fridley:2012fj,paquette2012,Pearson:2012uq} phenologies may aid the success of non-native species; while more recent work suggests that non-native species may be the major drivers of longer growing seasons in North America \citep{wolkovichAmBot2013}. % termchange: invasive to 'non-native;'

Here we build on current theoretical perspectives and empirical work to develop predictions of how phenology may enhance plant invasions with climate change. Our review begins first with the role of phenology in avoiding or mitigating disturbance, stress, and competition. Next, we review recent literature on plant functional traits to highlight evidence for a fundamental trade-off between flowering phenology and the return rate of growth investments, which may impact how climate change and phenology affect invasions. Considering projected scenarios of climate change we make mechanistic predictions for when during the growing season across three major ecosystem types vacant phenological niche space may promote invasions, and consider current evidence for our predictions. We close by reviewing major questions whose answers would improve predictions of future invasions via phenology.\\

% To date, however, work has focused mainly on documenting phenological differences between native and non-native species within invaded communities, with less effort applied to articulately framing and testing models where especially distinct (e.g., earlier or later than most of the native community) phenologies are a fulcrum for non-native species' success. This is a critical gap since connections to such theory would at once provide a framework to test theory from basic plant ecology and invasion biology, and link to a predictive framework of how phenology, alongside a suite of other plant traits, may shift with climate change. Such extensions would then provide the basis to begin to model how ecosystems themselves mays shift in the future with continued climate change. We suggest that progress at the intersection of invasion biology and plant phenology could be rapid, but would benefit from far more efforts to integrate studies into the frameworks provided by basic life history and plant strategy theories.\\
\noindent \emph{A note on terminology}\\
Given debate over terminology in invasion biology \citep{Colautti2009} we wish to be clear about our definitions. We use `non-native' to refer to any species established outside of its home range; such a distinction between native and non-native is important because non-native species have evolved in a different community than the one into which they have been introduced. Thus, they may exhibit differing strategies and trade-offs than native species.  We use the term `invasive' for non-native species with a detrimental impact in their introduced community \citep[following][]{Mack:2000bn}. Finally, `invade' and `invasion' refer to the introduction of species, whether they are eventually invasive or not.\\

\noindent {\bf Phenological strategies}\\

\noindent \emph{Phenology within plant life history theory}\\
\noindent Phenology is an important component of plant life history theory \citep{Almufti1977,Grime:1977sw,stanton2000}---affecting both biotic constraints (e.g., competition, herbivory, pollination) and abiotic constraints (e.g., frosts, drought) on plant performance. Extensive work over the past several decades has focused on how biotic interactions are informed by phenology, including competition \citep[e.g.,][]{Rathcke:1988yc,VANSCHAIK:1993uq} and mutualisms \citep[e.g.,][]{Brody:1997ro}, while recently the balance of studies has shifted towards a more abiotic focus with climate change \citep[e.g.,][]{Inouye:2008gj,Miller-Rushing:2008zv}.

Plant strategy theory has generally focused on how both abiotic and biotic factors affect acquisition, allocation and loss, often extrapolating into a focus on how well plants handle competition, stress and disturbance \citep{Grime:1977sw}. Stress, as generally defined, does not lead to major tissue loss while disturbance does---thus the best definition of stress versus disturbance is often species and location specific \citep{crainebook}. For example, in temperate systems, both frost (abiotic) and herbivores (biotic) may act as a disturbance around which plants must balance their leafout timing. The major difference between the plant's ability to adapt to these abiotic and biotic constraints arises via the feedbacks possible with biotic factors (e.g., herbivores may adjust, within their own set of climatic limits, to match earlier leafout). In contrast, abiotic constraints on plant growth (e.g. frost) are unlikely to be impacted by plant phenology.  Thus, in most systems where abiotic factors have been relatively stationary across years---in timing and variability especially---we expect plants to have adjusted their strategies to these system properties. Further, given temporal variation in the abiotic and biotic environment (i.e., across the growing season and across years), we expect phenology to be a major axis along which plants structure their overall life history strategies \citep{Grime:1977sw}. Indeed, recent research in the expanding field of functional plant traits suggests phenology may be tied to a suite of other traits producing several major phenological strategies.\\

\noindent \emph{Trait correlations with phenology}\\
\noindent A review of the functional traits literature (Table 1) highlights a strong axis for flowering phenology where earlier flowering is associated with a suite of traits related to rapid return on investment, while later flowering is often associated with the reverse. This axis makes sense when considering how stress, disturbance and competition vary across the growing season in many systems (Fig. 1): early in the season when abiotic stress and disturbance are high, but competition low, an early-flowering, rapid growth, and comparatively low-investment strategy allows species to grow and reproduce quickly before periods of strong competition begin. Such a strategy may also make some loss of tissue to environmental disturbance early in the season less detrimental if rapid growth allows rapid replacement of tissue. While it may seem obvious that earlier flowering would require a quicker return on investment, many perennial species use resources from previous years for current-year's flowering, and thus this correlation is not automatic \citep{muller1978}. Further, such a trade-off is seen across both herbaceous and woody species (Table 1). In contrast to early-flowering species, species that flower later in the season must survive high competition throughout the mid-season and thus traits that allow more efficient access to, transport and use of resources would be key. Loss of tissue to disturbance in such a strategy, however, may impart a relatively higher cost, as regrowth would be much slower. This major phenological trait axis---of early and fast versus later and slower---has been noted by many researchers \citep[e.g.,][]{Lechowicz:1984cr,Sun:2011eu}, but an additional strategy may be viable in the late season. It seems that some species may also exhibit a rapid growth and low investment strategy in the late season \citep{Sun:2011eu}, however, as this has been noted less often, we do not consider it extensively.
% Layered onto this, phenological traits themselves are often also correlated: species that leaf out early often also flower early \citep{phenophylo} and in temperate mesic system early-flowering species are those which track interannual climate variation most closely \citep{phenophylo}.

Examining phenology as one trait within a complex network of correlated traits raises an important issue of considering when phenology is a major trait on which selection acts, versus only linked to a more critical trait. For example, flowering time is often associated with seed size (Table 1) and teasing out how much phenology or seed size is constrained by the other remains a puzzle, with correlations varying by clade and study \citep{Mazer:1989in,Bolmgren:2008vo}. Thus, phenology may be structured strongly by selection on it directly, via links with other traits, or shaped by evolutionary history \citep[see][and see below]{Lechowicz:1984cr,Ollerton:1992kf}. If, however, phenology is a major trait structuring life history strategies, then given a relatively stationary abiotic and biotic environment we would expect each species to optimize its phenology for that environment. Given sufficient time and species dispersal we would also predict that communities would contain a suite of phenological strategies that take up most available resources across the growing season.\\

\noindent {\bf Climate change, phenology, and invasions: Predictions}\\
\noindent Climate change has altered the climate of most ecosystems globally such that they cannot be considered stationary \citep[that is, to have consistent climate means with some stochastic variation around those means, see][]{julio2012}. Non-stationarity could change the optimal phenological strategy---both in absolute timing, and in flexibility in this timing---leaving native species less well-adapted to their current environment and providing an opportunity for invasions. Understanding how phenology may intersect with plant invasions requires an explicit temporal model of how competition, stress and disturbance vary across growing seasons and with climate change. Such a model should make basic, but testable predictions about when---within a growing season and over longer timescales---in which systems, and how phenology may contribute to invader success. We lay out general predictions below, but focus on how temperature increases and precipitation change may affect species invasions. Given this focus we consider predominantly three types of systems which differ in the dominant abiotic controls over phenology---temperate mesic systems (temperature-control), temperate grasslands (temperature and precipitation control), and semi-arid systems (precipitation control). These systems provide useful contrasting examples of how competition, stress and disturbance may interact with phenology to influence community assembly and invasions. % Our considerations focus mainly on systems with one pronounced growing season each year, but we also consider systems with bimodal or less clearly defined growing seasons.

We base predictions on recent climate change projections \citep{knutti2013}, trait correlations and plant strategies related to phenology (Table 1 and above), and more generally to invasion. Specifically, we assume (1) that across space and time non-native species may invade environments of relatively high stress and disturbance, but low competition \citep[e.g.,][]{rej1996,Gelbard2003}, up to a point, (2) as there is also evidence that non-native plant species rarely occupy the most climatically stressful environments \citep{rej1989}. Additionally, (3) invaders are often more plastic and thus may adjust to new conditions quickly \citep{funk2008,Hierro:2009up,Davidson2011,wainwright2013}. We thus predict temporal opportunities for invasions in periods of relatively high stress and disturbance, but low competition, and discuss how and when climate change may alter these opportunities. \\

\noindent \emph{Predictions: Early-season}\\
\noindent  Across systems with distinct growing seasons, the early season often represents a period of relatively high stress and disturbance but low competition (Fig. 2)---as most species slowly begin to reactivate tissues and grow. Climatically, the early portion of a temperate growing season is signaled by a rise in temperature. This rise, however, is correlated strongly with increased variability in temperatures (Fig. 2), resulting in high stress and possible disturbance for plants active during this period, and---relatedly---low competition. Plant species active in the early season may risk tissue loss to frost damage or other extreme temperature swings present in the spring \citep{Linkosalo2000,Augspurger:2009gj}, or to greater herbivore damage \citep{Lechowicz:1984cr}, as species my be more apparent to herbivores in the early-season \citep{Brody:1997ro} and have less-defended tissues \citep{vanasch2007}. Early-season phenologies, however, also benefit from reduced competition. Across habitats, few species leaf and flower early (Fig. 2) yielding lower competition for soil and light resources and for pollinators \citep{Mosquin1971}. Additionally, competition with microbes for soil nutrients may be especially low as in many systems the soil microbial community turns over with warm spring weather, producing a flush of soil nutrients \citep{Zak:1990ar}. 

% Below, termchange, switched from invasive to 'First, non-native'
Early phenologies of non-native species may thus succeed through two major mechanisms independent of climate change. First, non-native species that are active early in the growing season may be particularly successful because they have the opportunity to preempt space and soil resources, grow quickly and shade later-active species \citep{Weiner1990,Wilsey:2011cr,wolkovich:2010fee}. This type of asymmetric competition could create ``seasonal priority effects,'' one mechanism by which non-native species could establish and rise to dominance in a new community \citep{dickson2012,Wainwright:2012tw}. Second, invaders may succeed via an early-season enemy release mechanism. In invasion biology the enemy release hypothesis suggests that non-native species could be less susceptible to herbivory than native species due to a lack of specialist herbivores \citep{Keane:2002uz,Liu:2006kj}. Thus, if the early season is a critical period of susceptibility to herbivory \citep{Feeny1970,Barbehenn2013} this may also be the critical time for invaders to benefit from herbivore release. This suggests a mechanism by which non-native species could break the risk-benefit trade-off of early phenology experienced by native species, thereby giving them a special advantage from early-season phenology.

Climate change may additionally promote early-season phenologies and provide a mechanism whereby non-native species with early-season phenologies are especially successful. Recent increases in spring temperatures in the temperate biome \citep[at least partially associated with increases in greenhouse gases, see][]{Trenberth:2007hk} have been studied extensively. Most studies find that spring temperatures have increased as much or more than temperatures in other seasons \citep{cohen2012}---meaning this is a season of especially high non-stationarity in climate (Fig. 1). Predictions for precipitation-limited systems are more variable but include options for increased total and increased variability in early-season rainfall \citep{Trenberth:2007hk,knutti2013}. Such high variability may make it an optimum period for invasion of other species for several reasons. First, such high non-stationarity should mean native species are being pushed far away from the previous long-term climate means to which their phenology should be adapted. Next, if non-native species have higher phenological plasticity \citep[e.g.,][]{wainwright2013} they may more closely track shifting climate than native species. Early-flowering is also often correlated with plant traits related to rapid growth. Non-native species exploiting an early-season vacant niche may thus grow rapidly and take up much of their needed resources to complete reproduction before competition with native species effectively sets in (Fig. 3). \\

\noindent \emph{Predictions: Mid-season}\\
\noindent In most systems the mid-season represents the period when the greatest number of species flower \citep[e.g.,][]{Fitter:2002sm,Aldridge:2011}, and work to date using community datasets suggests that this is true for both native and non-native species \citep{wolkovichAmBot2013}. This is perhaps not surprising as in most systems the mid-season represents a period of relatively low stress due to the physical environment and high abundance of pollinators. This high abundance of species in flower, however, means it is also the period of high competition for resources (Fig. 2) and thus we generally predict few invasions driven by phenology mid-season (Fig. 3). 

In some systems with precipitation-control, however, the mid-season often has a reduced period of plant competition associated with a mid-season drought (Fig. 2). Temperate grasslands, for example, are generally characterized by mid-season droughts when the highest temperatures coincide with low precipitation, drying soils, and fewer species that initiate flowering during this period \citep[e.g.,][]{Craine:2012kl}. This reduced competition could result in an opportunity for invasion. However, because it also generally represents a period of extremely high drought stress, we do not predict invasions mid-season generally. Increases in mid-season precipitation---especially those outside historical ranges---may, however, provide a novel vacant niche. Many early-flowering species appear to end flowering well before drought onset and, depending on the phenological cues they use for flowering onset and end, may thus not be able to exploit greater mid-season precipitation because they have adapted primarily to avoid the mid-season drought \citep{Craine:2012kl,Craine2013}. Native species flowering during the mid-season may have trade-offs between drought tolerance and competitive abilities for other resources \citep{Craine2013}, which may make them less successful at exploiting increased mid-season precipitation. We speculate that, under this scenario, increased mid-season precipitation (that is not offset by higher temperatures) may promote mid-season invaders with climate change.\\

\noindent \emph{Predictions: Late-season}\\
\noindent Climatically, in many systems the end of the growing season mirrors the beginning (Fig. 2); but plant phenology differs strongly. For example, while in both temperate mesic and grassland systems plant leafing and flowering (Fig. 2) generally tracks closely the rise in spring temperatures around 5$\,^{\circ}\mathrm{C}$  \citep{larcher2003}, community flowering curves generally end at least one month before temperatures return to 5$\,^{\circ}\mathrm{C}$. The reasons for this may be simply related to physiological constraints: species must flower well before the end of the season in order to have enough time to fruit and set seed. A balance between risk and investment may also be important: as mean temperatures drop, temperature variability climbs (Fig. 2), just as low spring temperatures are also correlated with higher temperature variability. In contrast to the spring, however, later in the season almost all species have made a heavy investment in growth and reproductive tissues and loss to frost may pose an even greater threat than frosts in the spring, thus plants may be especially conservative, even those with a `late and fast' phenological-growth strategy. This risk/investment balance may explain why many species often have plastic spring phenologies based on temperatures that are flexible across habitats, but static fall phenologies based mainly on photoperiod cues \citep{Howe:2003}. The result is that, in many systems, the end of the growing season is a period of generally climatically high stress and disturbance, but low competition. Without climate change we expect most native species have adapted towards the optimal time to senesce based on their risk-investment strategy and there may be little opportunity for invasion.

With climate change, however, we expect pronounced opportunities for invasion late in the growing season when climate change extends the end of the season (Fig. 3), as native species may be constrained by their evolutionary history, and cues associated with the end of the growing season (i.e. photoperiod). Thus, non-native species with greater phenological niche breadth (either via a generally static longer growth or flowering period, or via greater flexibility to extend their phenology resulting in greater niche breadth) may be able to exploit this late season vacant niche. This should apply to all systems where the late season is a period of relatively high stress and disturbance, but low competition---up to a point. In many systems where both temperature and precipitation shape growing season dynamics the late season can have high drought stress; we do not expect significant invasion during this window because non-native species may not be as adapted to high drought stress compared to the native species in these systems \citep{alpert2000}.\\

\noindent {\bf The role of climate change in phenologically-mediated invasions: Evidence to date}\\
% We review stuff here. It's brief. There are little data on some stuff (precip changes probably). We also discuss where more research is critically needed.\\ Recent increases in spring temperatures in the temperate biome \citep[at least partially associated with increased in greenhouse gases, see][]{Trenberth:2007hk} are the most studied climate change effects on plant phenology \citep{Pau:2011wd}, yielding the most estimates of how species have responded \citep{Parmesan:2003dm,Root:2003kl,Menzel:2006sq}. ADD IN SOME LINKS to figure.\\

\noindent \emph{Results: Early-season}\\
\noindent  Work to date supports evidence for early-season invasions, which are linked to climate change in several temperate mesic systems \citep{wolkovichAmBot2013}, and linked to seasonal priority effects in semi-arid systems \citep{dickson2012,Wainwright:2012tw}. Additionally, research on one non-native understory shrub species \citep{Xu:2007he}, suggests a role for seasonal priority effects in temperate forest systems. Results in temperate grasslands, however, do not show a strong link between early phenology and invasion \citep{wolkovichAmBot2013}. Across temperature-controlled systems, however, early-season species, whether native or non-native, also tend to be the most sensitive to temperature \citep{Cook:2012,Wolkovich:2012n}. Relatedly, multiple studies using various methods now show that in many mesic temperate biomes, invaders are highly sensitive to temperature---tending to advance their phenology significantly more than native species \citep{Willis:2010al,wolkovichAmBot2013}, though, again, this does not appear to be the case in temperate grasslands \citep{wolkovichAmBot2013}. Moving forward, accurate predictions of phenologically-mediated invasions will require teasing out exact mechanisms. Thus, future research with climate change in temperate biomes will need to quantify how much invasion success occurs because of flexibility in phenology (i.e., the trait of the invader leads to success) versus via open niche opportunities present in the early-season due to non-stationarity with climate change (i.e., the system is open to invasion in the early-season), or a mix of the two scenarios. \\

\noindent \emph{Results: Mid-season}\\
\noindent  Several North American studies have documented declines in native species alongside shifts with climate change in mid-season drought periods---generally showing that the drought period may be extending or becoming more pronounced \citep{Aldridge:2011}, and multiple authors have postulated that this period will result in a vacant niche for invaders \citep{Sherry:2007fq,Aldridge:2011}. To date, however, no studies (of which we are aware) have shown invaders occupying these mid-season drought periods, while in contrast two studies of phenology at Konza Prairie LTER have found a decline in non-native flowering coinciding with the mid-season drought \citep{Craine:2012kl,wolkovichAmBot2013}. Further, work on previous extreme droughts has generally documented a shift in native species composition, but not invasions \citep{Weaver1936}. These patterns are based on findings in North American prairie systems, however; far more work is needed additionally to understand if this period is occupied by invaders in other systems or is possibly too stressful. This represents an area where predictions are difficult for several reasons: (1) understanding and modeling how species respond to moisture has proven far more difficult than modeling temperature responses \citep{Crimmins:2011lq,wolkovichAmBot2013}, (2) work to date suggests phenological responses to drought are highly variable between different species \citep{Jentsch:2009ff,Prieto:2009fu}, and (3) projections of how precipitation and droughts will shift in the future are some of the most uncertain of all climate change forecasts \citep{knutti2013}. 
\\

\noindent \emph{Results: Late-season}\\ %termchange -- added 'and invasive' to first sentence
\noindent Recent studies of plant invasions, especially in eastern North American forests, suggest non-native and invasive species may successfully exploit a late season vacant niche via greater niche breadth in temperate biomes---which may be linked to climate change. A pervasive non-native understory species in the Ohio River Valley, \emph{Lonicera maacki}, stays green later than any native understory species \citep{becker2013}. Similarly, the invasive tree \emph{Acer platanoides} also stays green later than one studied North American congener \citep{paquette2012}. Further, recent work suggests these non-native species may not play by the same risk-investment rules as native species in the fall. A study of several dozen North American non-native species from the understory confirms that these species consistently senesce in the fall later than native species \citep{Fridley:2012fj}. \citet{Fridley:2012fj} also found that many of these species remain green until the first major frost and thus lose their leaves to frost versus plant-induced senescence. This should be a major cost to the plant---as most species resorb nutrients well before the first frost \citep{Lambers:2008jb}. Further work showed that the longer leaf life-span of the non-native species enabled a greater time-integrated nutrient-use efficiency \citep{Heberling2013}, and, coupled with high rates of nutrient uptake the following spring, provides a mechanism by which the unique phenology of non-native species, compared to the native community, could promote invasion.

If these non-native species in eastern North America do gain a large benefit from occupying an open late-season temporal niche, even while losing tissue to frost, then climate change could further increase their success. In many habitats, fall temperatures are rising more quickly than even spring temperatures \citep{cohen2012}, and are expected to continue to rise---further extending the open niche space at the end of the season, which already appears temporally much greater than the early season (Fig. 2). Thus, the late-season appears to be a period of very low plant competition and is often also when microbial communities turnover \citep{Bardgett:2005ls}---suggesting that species which can remain active until the end of the season may have access to a large available resource pool. 

Across systems, however, autumnal shifts in climate and phenology are still relatively unstudied compared to spring. More work is needed to understand where, and by how much, mean fall temperatures are shifting in comparison to other seasonal temperature shifts, and how species are adjusting their late-season phenological events. In particular, there is very little work from the late-season in grasslands, where the combination of shifts in mid-season droughts alongside shifted fall temperatures may create novel climates and, possibly, novel opportunities for invasion. Alternatively, systems with mid-season droughts may have reduced opportunities for invasion if such droughts---over the long-term---have favored more variable phenological strategies \citep[e.g., species flower very late to avoid stress of mid-season drought, see][]{Craine:2012eco} compared to simple temperature-controlled systems, which have been noted to have far greater late season empty phenological space compared to grasslands \citep{Craine:2012eco}. \\

\noindent{\bf Major questions}\\
Recent research connecting phenology and invasions has clearly provided support for the idea that invaders may benefit from phenological vacant niches during periods of relatively high stress and disturbance but low competition. Much further work is needed, however, to mechanistically link phenology to plant invasions and to build toward more accurate and specific predictions of how climate change may promote invaders in the future. Below we outline what we consider as the major questions impeding robust predictions in this area.\\

\noindent \emph{How do longer term properties of a system's climate affect phenological invasions?} \\
\noindent  Climate change represents a long-term climate trend overlaid onto already complex climates of most ecosystems--including a climate's mean, cyclical variation (e.g., seasons and multi-year cycles often driven by large-scale oscillations) and extremes. Thus, a coherent understanding of how species and communities will shift with climate change requires consideration of more than shifts in the timing and magnitude of temperature and precipitation. 

A coherent picture of how species respond to shifts in temperatures and precipitation will include a focus not just on mean or aggregated climate metrics, but also on extreme events. An example of the importance of this comes from attempts to understand the correlation between phenology and performance with climate change. Several studies show that species that tend to advance with warming also tend to increase in abundance or performance \citep{Cleland:2012vn}, including invasive species \citep{Willis:2010al,chuine2013}; yet other work shows that the early season (native) species most sensitive to climate are those that suffer the greatest performance losses with climate change \citep{Inouye:2008gj}. Such conflicting results can be better understood when considering the role of extreme events. In the latter study, a shift in earlier springs that did not coincide with a shift in frost dates produced the performance declines \citep{Inouye:2008gj}. Accurate predictions of which systems may have viable early-season open niches for invasions will require examining how much temperatures have shifted on average while also considering important spring climate events related to plant stress and disturbance, such as frost. To date, increases in frost risk with spring warming have been documented in parts North America \citep{Inouye:2008gj,gu2008,Augspurger2013}, but not much of North America \citep{Easterling:2000sa}, nor in Europe \citep{Menzel2003a,Scheifinger2003} or China \citep{Dai2013}, where shifts in freeze and frost days have occurred in step with the climate shifts driving earlier spring onset \citep{Dai2013}. Currently, there is little known about how early-season non-native versus native species cope with frost and frost risks. Additionally, most work on phenology has focused on temperature and related events such as frost, with little work on precipitation events.

Studies of precipitation-controlled systems must also deal with large-scale, longer-term cycles in precipitation that often dominate in such systems (e.g., El Ni\~{n}o in many semi-arid systems in western North America). Such longer-term cycles may be a critical consideration because native species may be adapted to these cycles. Relevant studies of population dynamics will, therefore, need to work across the relevant climate oscillation timescales and predictions will need to consider whether the oscillation may shift with climate change, as projected \citep{ipccPhys2007}. Such oscillations may also be directly important to non-native species as they may dictate invasion lags and jumps \citep{Salo:2005eo}. \\
% Indeed, precipitation-controlled systems overall are the ones in which predicting how climate will shift in the future has proven most difficult \citep{knutti2013}.

\noindent \emph{Do invaders share the same strategies \& trade-offs with phenology as native species?}\\
\noindent  Our predictions here are based on the plant traits literature that suggests several major options for species' phenological strategies---and thus how phenological traits may trade off with other traits. Thus, a critical assumption of our hypotheses is that these strategies are consistent across non-native and native species. It is possible, however, that non-native species may exhibit novel strategies, similar strategies that involve additional traits, or that they may exhibit trade-offs of different magnitudes.

Understanding these trade-offs more fully has additional predictive benefits. In particular, it would advance efforts to integrate phenology within a more holistic trait framework and enable scaling to ecosystem predictions more easily. If certain phenological strategies are more common in invaders, it would suggest a suite of plant traits that would change in concert with plant invasions and climate change. Recent work shows that temperate non-native species are more phenologically responsive to temperature than native species at some sites \citep{wolkovichAmBot2013} and that species that advance with warming also tend to increase in abundance and performance \citep{Cleland:2012vn}. Together these findings suggest that future temperate ecosystems may be dominated by more phenologically plastic species. If such flexibility correlates with other traits (e.g., lower leaf lifespan or lower nutrient content in leaves) we would then predict cascading shifts in ecosystem properties such as decomposition and nutrient cycling. \\

\noindent \emph{When and how does high phenological flexibility yield a competitive advantage?}\\
\noindent  One commonality across climatically diverse systems is evidence linking non-native species with high phenological flexibility. Across systems where the start of the growing season is determined by temperature \citep{wolkovichAmBot2013} or by precipitation \citep{Wainwright:2012tw} species that track the start of season most closely are highly successful invaders. Understanding the benefits and trade-offs of high phenological tracking of environmental variables would address a fundamental question in invasion biology: if high plasticity yields a competitive advantage, why do species differ in their plasticity? One hypothesis is that species that track climate change well over the timescales for which we have data (or focus our analyses) may suffer major population losses during years of extreme climate. Alternatively, the current non-stationarity of climate may have shifted the playing field; this may mean species that evolved in more variable environments are now often successful invaders. These hypotheses have not been tested to our knowledge. However, adaptations of bet-hedging models \citep[e.g.,][]{donald2013}, combined with currently-available climate data should allow basic vetting. 

Studies examining the plasticity of phenology in non-native species may also want to consider how much evolutionary change following introduction has contributed to this plasticity \citep[e.g.,][]{sultan2013}, and how quickly species can genotypically adjust to more static phenological cues post-invasion. Many of the studies mentioned here examine phenological shifts that are attributable to phenotypic plasticity (e.g., they are of marked perennial individuals or come from woody species known to shift leafing and flowering plastically between years with different climates), but recent studies have documented rapid evolutionary shifts in invaders, especially in phenology \citep[e.g.,][]{Colautti:2011,konarz2012,novy2012}. Understanding how much phenological flexibility is driven by underlying plastic versus genetic shifts is important to projections of which species may become invaders---if much change is genotypic it suggests then that predictions may be more difficult and will require knowing \emph{a priori} how quickly phenology can evolve under new selection regimes. More studies examining invaders in their native and introduced ranges \citep[e.g.,][]{Godoy:2009dz,matesanz2012} would begin to build data on how often phenologies shift with invasions and how similar or distinct invader phenologies are in their native versus introduced ranges \citep[e.g.,][]{wolkovichAmBot2013}.\\

\noindent \emph{How does evolutionary history influence phenological invasions?}\\ 
\noindent  Research from molecular ecology has consistently shown a strong genetic basis for leafing and flowering times within species \citep{Howe:2003,arabid2011}, thus it may be expected that related species would share similarities in their phenologies, and possibly their phenological responses to climate. Indeed, a growing number of studies have documented significant evolutionary structure in the distribution of flowering times and sensitivities to climate change across species \citep{Davis:2010ls}, including invaders \citep{Willis:2010al,wolkovichAmBot2013}. A recent, more comprehensive analysis across \(>20\) sites, from temperate to tropical Northern Hemisphere zones, shows phylogenetic structure in flowering and leafing for almost all communities studied \citep{phenophylo}, such that more closely related species also tend to have more similar phenologies. This structure means studies of phenology including multiple species may want to consider how much variation in phenology and related phenological traits are explained by the evolutionary distances separating species, versus other factors.

Considering phylogentic structure is especially important in studies attempting to link phenology to invasion success and any studies looking for correlations between phenology and other traits, because species cannot be treated as statistically independent. For example, studies finding multiple non-native species with distinct phenologies compared to the native community will need to test how much this finding is driven by the phylogenetic affinity of non-native species compared to species in the native community. If non-native species are only distantly related to the native species pool, we might expect them to differ in their phenologies, irrespective of the actual traits explaining their invasion success. In addition, when looking at correlations between phenological traits and invasion, apparent trade-offs might simply reflect phylogenetic affinities if non-native species are evolutionarily distant for the native species pool. Our mechanistic explanation for invader success might, thus, be quite different depending on whether invaders are filtered on phylogentically conserved traits versus a scenario in which they diverge from the native community following introduction. Phylogenetic methods aid in distinguishing between these two scenarios. \\

\noindent \emph{Conclusions}\\
With future climate change invasive species are predicted to increase both in abundance and in spatial distribution \citep{ipcc2007summary,bradley2010}. We have outlined here a more focused framework for examining how phenology may affect plant invasions. This framework considers phenology as one factor by which plants attempt to optimally balance acquisition, allocation and loss in an environment where most systems' climates are now highly non-stationary. As increasing research builds to test and advance this framework, resource managers will in turn need to evaluate how they may use phenology in their decision-making. If many species appear to gain a foothold or spread in introduced communities via phenology it may suggest novel management practices including which species may have a high potential to be invaders with climate change and when the best time to apply treatments may be. Such applications will, of course, be bolstered by additional studies of phenology. In particular, further work is needed to understand how phenology correlates with and is constrained by other traits, whether this varies between different climate regimes, functional groups and clades, and whether non-native species appear to face the same constraints to their phenologies as native species. \\

\noindent {\bf Acknowledgments}\\
Comments from J. Craine, J. Dukes, B. Cook, T. J. Davies, and two anonymous reviewers improved this manuscript. EMW was supported by the NSERC CREATE training program in biodiversity research. 
\newpage
\bibliography{/Users/Lizzie/Documents/EndnoteRelated/Bibtex/LizzieMainMinimal}
\newpage
\begin{center}  
\begin{table}
\caption{Current research suggests one major axis by which phenology co-varies with other traits: earlier-flowering (and in some cases, earlier-leafing) is often associated with traits related to quicker returns on investments (faster growth rates, shallower roots, etc.) while later-flowering species show traits associated with slower returns on investments (slower growth rates, greater heights, deeper roots etc.). Studies characterizing this trade-off are presented above the double-line, while additional studies are shown below. We reviewed the literature using ISI Web of Science and the following search: Topic=(phenolog*) AND Topic=(plant*) AND (functional trait*)
Refined by: Document Types=( ARTICLE ) AND Web of Science Categories=( ECOLOGY ), which returned 79 papers. Of these we included studies that documented phenology and at least one other trait for multiple species. Studies were excluded if they only studied animal guilds or if they focused on selection within a single plant species. Additionally, leaf lifespan was omitted as a measure of phenology if it was simple evergreen/deciduous (as this represents more leaf lifespan than phenology). We included several additional studies that we encountered in the process of writing the manuscript. `Flowering date' includes flowering date, peak flowering date and flowering onset date; SLA=specific leaf area. {\bf Continued on next page.}}
\vspace{1.5 pt}
%   \begin{tabular}{ | p{2.5cm}  | p{7.5cm} | p{5cm} | }  
% \hline 
% & Focus & Scales  \\ \hline
% \end{tabular}
\begin{minipage}{16cm}
\begin{tabular}{ | p{2cm} |  p{2.5 cm} | p{2.5 cm} | p{2.5 cm} | p{5cm} | }  \hline
Phenological trait & Other trait(s) studied  & Relationship & Plant functional group(s) & Location(s) and reference(s) \\ \hline \hline
flowering date	& max height & positive (earlier, smaller) & herbaceous species	& Mediterranean old field in Israel \citep{hadar1999}; Semi-natural grasslands in France \citep{Loualt2005,Vile:2006nc}; Southeastern Sweden \cite{Bolmgren:2008vo}; Tibetan Plateau \citep{du2010}; Eastern North America \citep{Sun:2011eu}; Mountain meadows in Italy \citep{Catorci2012} \\ \hline
flowering date	& max height & positive (later, taller) & mixed woody and herbaceous & Ponderosa pine forest \citep{Laughlin2010} \\ \hline
flowering date & seed size & positive (later, larger) & herbaceous species & semi-natural grassland in France \citep{Vile:2006nc} \\ \hline
flowering date & seed size & negative (earlier, smaller) & mixed woody and herbaceous & Indiana (USA) dunes \citep{Mazer:1989in} \\ \hline
flowering date	& growth rate & negative (earlier, faster)	& herbaceous species & Eastern North America \citep{Sun:2011eu} \\ \hline
flowering season & rooting depth & positive (later, deeper) & herbaceous species & Patagonian Steppe \citep{Golluscio:1993xi}; Mediterranean annual grassland, California, USA \citep{Gulmon1983}; Tibetan Plateau \citep{Dorji2013} \\ \hline
length of growing season & rooting depth & positive (later, deeper) & mixed & Semi-arid woodland in Australia \citep{Campanella2008} \\ \hline
\end{tabular}
\vspace{-0.75\skip\footins}
   \renewcommand{\footnoterule}{}
  \end{minipage}
\end{table}
\end{center}

\begin{center}  
\begin{table}
\vspace{1.5 pt}
%   \begin{tabular}{ | p{2.5cm}  | p{7.5cm} | p{5cm} | }  
% \hline 
% & Focus & Scales  \\ \hline
% \end{tabular}
\begin{minipage}{16cm}
\begin{tabular}{ | p{2cm} |  p{2.5 cm} | p{2.5 cm} | p{2.5 cm} | p{5cm} | }  \hline
Phenological trait & Other trait(s) studied  & Relationship & Plant functional group(s) & Location(s) and reference(s) \\ \hline \hline
flowering date	& generation time & positive (later, longer) & herbaceous species & Semi-natural grassland in France \citep{Vile:2006nc} \\ \hline
flowering date	& SLA & negative (earlier, thinner) & herbaceous species & Semi-natural grassland in France \citep{Vile:2006nc} \\ \hline
length of growing season & SLA & negative (longer, thicker) & mixed & Semi-arid woodland in Australia \citep{Campanella2008} \\ \hline
leafout date & diameter of spring-vessels and/or or greater numbers of narrow-diameter vessels & positive (later, larger)  & trees & Northern North American forests \citep{Lechowicz:1984cr}\\ \hline
flowering date & leaf tissue density & positive (later, greater) & herbaceous species & Tallgrass prairie in Kansas, USA \citep{Craine:2012kl}\\ \hline
flowering date & grazing tolerance & negative (later, tolerant) & herbaceous species & Mediterranean old field in Israel \citep{hadar1999}  \\ \hline \hline
leaf flushing date & SLA & positive (later, thinner) & trees & Savannah/Cerrado in Brazil \citep{Rossatto2009} \\ \hline
length of leaf season & leaf size & positive (later, larger) & mixed woody and long-lived perennial species & high elevation Mediterranean woodland, Morrocco \citep{Navarro2010}\\ \hline
flowering date & seed size & negative (later, smaller) & herbaceous species & Southeastern Sweden \citep{Bolmgren:2008vo}\\ \hline
flowering date & seed size & bimodal (early and late flowering had small seeds, mid-season was mixed) & herbaceous species & Mountain meadows in Italy \citep{Catorci2012} \\ \hline	
mixed & morphology, leaf thickness, photosynthetic pathway, life history, seed biology & complex (multivariate) & mixed woody and herbaceous & semi-arid woodland, Australia \citep{Leishman1992}; Northeast China \citep{Wang2005} \\ \hline
length of growing season & origin & invading species had longer, later growing seasons & mixed woody and herbaceous & Germany \citep{Kuester2010}\\ \hline \hline
\end{tabular}
\vspace{-0.75\skip\footins}
   \renewcommand{\footnoterule}{}
  \end{minipage}
\end{table}
\end{center}


\newpage
\noindent 
\begin{figure}[h!]
\centering
\noindent \includegraphics[width=0.8\textwidth]{/Users/Lizzie/Documents/git/manuscripts/invasionphenology/figures/invasion_niches.png}
\caption{Basic invasion theory, built on limiting similarity theory, suggests that species should invade during times when most other species are inactive \citep[vacant phenological niche, see][]{wolkovich:2010fee}. Here we show idealized niche diagrams for four non-native species (purple, dashed-line distributions) and seven native species (gray distributions) in a hypothesized simple mesic temperate system where temperature limits viable periods for plant growth. Across the growing season variation in stress, disturbance and competition may dictate the optimal phenological strategy: with early-active and late-active species experiencing lower competition but also more variable temperatures, in the mid-season community flowering peaks (see Fig. 2) and thus we expect mid-season active species may be strong competitors for many resources.  With climate change extending viable periods for plant growth (dark blue lines) non-native species with highly plastic phenologies may have an increased opportunity for invasion at the start and end of the growing seasons in temperate mesic systems. As reviewed in \citet{wolkovich:2010fee} there are several major ways that species may exploit such vacant phenological niches. Species that can track the start of the season closely may exploit even very small vacant niches in the early season via priority effects. Additionally, climate change---by extending growing seasons in many systems---may increase vacant niche space at the start and end of the growing season, possibly allowing for invasions early and late in the season. Non-native species with early phenology and rapid growth strategies may succeed either early or late in the season, while species with greater phenological niche breadth (e.g., longer flowering period) may succeed late in the season.}
\end{figure}

\newpage
\begin{figure}[h!]
\centering \includegraphics[width=1\textwidth]{/Users/Lizzie/Documents/git/manuscripts/invasionphenology/figures/invasionphen_natexocurves_V2.png}
\caption{Flowering of non-native and native species within a community varies across habitats. This variation in flowering patterns may be strongly driven by climate differences between systems, which impact the various flavors of stress, disturbance and competition that plants experience. Mesic temperate systems such as Chinnor (UK, left) are often defined by temperature (darker blue shading), while other systems such as the tallgrass prairie of Konza (Kansas, USA, right) may have variable drivers across the growing season. In both systems temperature sets the beginning and end of the season and, as such, early-season species show strong sensitivity to temperature \citep{Cook:2012,Craine:2012kl}. In Konza, a consistent mid-season drought, however, coincides with a decrease in the number of species flowering at that time \citep{Craine:2012kl}. We assume that temperature below \(5^{\circ}\mathrm{C}\) limits development, as this is the temperature at which many cell processes slow dramatically or stop \citep{larcher2003} and further, is the suggested lower threshold temperature for tissue growth and development globally for
alpine trees \citep{Korner:1998qf}. Standard deviation (SD)
of temperature and the coefficient of variation (CV) of precipitation are given as pentad (5-day)
averages. Flowering data are species averages from NECTAR \citep{nectar}, climate data for Chinnor were taken from GHCN UK000056225 and cover 1853-2001, while climate data for Konza were taken from GHCN USC00144972 and cover 1893-2010.} %Standard deviation (SD) of temperature and the coefficient of variation (CV) of precipitation are given as pentad (5-day) averages. 
\end{figure}

\newpage
\begin{figure}[h!]
\centering \includegraphics[width=0.8\textwidth]{/Users/Lizzie/Documents/git/manuscripts/invasionphenology/figures/invasionphen_predictsV2.png}
\caption{Predictions for how climate change may promote invasions varies across the growing season, across systems with differing dominant climate regimes, and by how climate shifts (red arrows refer to temperature increases, while blue arrows refer to precipitation increases or decreases). Here we consider three major systems and how dominant climate drivers are projected to shift with climate change, based on recent models \citep{knutti2013}. Because models of precipitation shifts are often divergent \citep{knutti2013}, for systems with precipitation control we consider increases or decreases in precipitation. In all systems an increase in the dominant climate factor that controls the start of season may increase invasions by species that can track this shift closely (invader plasticity). Because we suggest that successful invasions are rare in times of very high resource competition and extremely high climatic stress and disturbance we do not predict invasions during periods when competition is already high (mid-season of many systems) or when climate change increases drought stress (declines in precipitation in semi-arid systems or during mid-season drought in grasslands). When climate change pushes systems far beyond their historic climate regimes, however, native species may be pushed well away from their optimal climate, and we may see an increase in invasions. See the main text for further details, including background assumptions leading to predictions.}
\end{figure}

\end{document}




\noindent \emph{How do phenological cues vary between non-native and native species?}\\
\noindent  Data are needed on exactly how non-native species are able to track the start of seasons more closely---that is, the exact complex of cues they use to trigger leafing and flowering. Fully identifying phenological cues for any species can be a long process, as many species appear to have complexes of cues involving temperature, soil moisture and photoperiod \citep{Stinchcombe:2004ec,Wilczek:2010ad}. It is, however, necessary to predict exactly which climate scenarios will trigger growth and thus susceptibility to early-season variability. For example, an non-native species may appear to respond only to temperature based on long-term field observations, but may actually only integrate temperature under certain photoperiods (which could be identified via a suite of lab studies varying both temperature and photoperiod). Native species may similarly respond to both temperature and photoperiod, but the exact mix of these cues may differ between the native and non-native species, making the non-native species able to track current climate patterns more closely. In addition to improving understanding of how non-native species may track the start of seasons more closely than native species, documenting exact cues may improve predictions from which climates future invaders will come, as trends in photoperiod and temperature cues are often predictably clinal \citep{Howe:2003,Wilczek:2009oa}. \\
%For example, returning to the example of photoperiod and temperature cues, non-native species in temperate systems may be expected to generally come from more southerly and more mild climates where photoperiod cues allow species to leaf and flower earlier than more northerly and climatically extreme locations.

% used to go after donald2013 (bet-hedging ref)
In particular long-term climate data could provide insight on how high population losses from extreme events must be, and how frequent such events inducing population losses must be to make high phenological tracking a detrimental long-term strategy. Similarly, climate data could form the basis to examine if non-stationarity has significantly shifted the playing fields to make high-phenological tracking the optimal strategy. Such models, however, will only provide basic tests of these hypotheses and require additional data for any useful predictions.

Progress will also come from understanding more quantitatively the risk/benefit of highly plastic strategies for non-native species \citep[e.g., non-native species that senesce later in the fall often lose green leaves to the first frost, which should be a major carbon and nutrient loss, see][]{Fridley:2012fj}. Testing the hypotheses laid out here requires knowing how much tissue gain and loss non-native species suffer across varying climate regimes they experience in their introduced habitats (i.e., across the variability in climate across the growing season and between years), especially in comparison to other species. Identifying what causes shifts in gains and losses---for example losses due to climate stress versus herbivory---will be critical to tying phenology to invasion success. [Could cut this paragraph?]
%



As climate change is predicted to effectively extend temperature-limited growing seasons by extending the period when plant growth is climatically-viable in the early and late season (Fig. 1). We suggest this extension of the seasons will lead to opportunities for invasions early and late season in systems with only temperature control (e.g., mesic temperate forests and old fields), but not in the middle.\\


Similarly, traditional theories in invasion biology have focused heavily on biotic interactions---in particular competition between functionally similar species \citep[limiting similarity and competitive exclusion,][]{macarthur1967,abrams1983}, or interactions with consumers and pathogens \citep[enemy release,][]{Keane:2002uz,Liu:2006kj}.\\

\noindent \emph{Data needs}
To date, work in semi-arid systems suggests invaders may occupy especially early-season niches, but such studies are still relatively rare compared to work in temperate systems. Additionally, there is a paucity of work on phenology and invasions in ecosystems where the start of the season is controlled by snowmelt, but such systems are known to have shifted greatly in many areas with climate change \citep[e.g.,][]{pederson2011}. For all such systems a major need for research is more data and study of relevant climate variables especially soil moisture, soil temperature and snow cover---variables that, compared to temperature, are few and far between. These types of data, however, we expect will prove key in improving understanding of phenology in precipitation controlled systems, including alpine, grassland, semi-arid and arid systems.  \\


% and understanding how these species may actualy benefit from this strategy could help improve understanding of the mechanism by which a unique phenology, compared to the native community, may promote their invasion.

\noindent {\bf Links to traits and ecosystems}\\
\\
Examining plant invasions and phenology from the perspective of how plants balance acquisition, allocation and loss amidst a changing playing field of stress, disturbance and competition has many research benefits. It integrates applied questions within a framework that allows them to advance basic and applied biology at once. It ties more directly to mechanisms, which allows findings to be more easily ported to other relevant systems. We see two benefits, however, as particularly important for research related to phenology: (1)  it forces phenology to be examined from the perspective as one trait within a network of correlated traits and, relatedly, (2) it allows more rapid scaling of how invasive species will affect ecosystems. We discuss the value of each of these in turn.\\

In a similar vein of the need to consider phenology as just one of many plant traits in an individual's strategy to balance acquisition, allocation and loss, more research is needed that examines links between phenological events. To date much research has focused on flowering, and, secondly leafing---which may be tightly linked in many species (e.g., earlier leafing allows for earlier flowering), with fewer studies still on fall coloring and senescence and very few on fruiting and seeding. This singular event focus is not surprising as often data are only collected on one event. By their very nature, of course, phenological events are each coupled to one another in sequence, which means that examining only one provides only a small sliver of the complete temporal story. \\

To date, most work on invasions and plant phenology has tended to focus rather fixedly on phenology as a single trait of interest \citep[but see][]{Sun:2011eu,hahn2012}, yet plant strategy theory \citep{crainebook} highlights how phenology is just one in a suite of traits related to strategy. For example, species with low leaf lifespans often also have higher leaf mass areas, suggesting a relatively low investment in leaves and possibly ruderal strategy \citep{Mack:1996ly}. Evidence to date suggests phenology is probably also correlated with other traits, such as plant height, seed mass \citep{Bolmgren:2008vo} and growth rate \citep{Sun:2011eu} and, for temperate trees, traits related to plant hydraulics \citep{Lechowicz:1984cr}. In this latter case, later leaf-out phenologies were associated with wider vessels and/or a greater number of narrow vessels in their wood, compared to native species. Larger vessels can transport water more rapidly but also are more susceptible to breakage during more extreme climate events (e.g., frost or drought). Because many trees must activate their hydraulics systems before they can leaf out each spring, this leads to the hypothesis \citep{Lechowicz:1984cr}
that later phenology may be an enforced requirement for species with larger vessels and highlights how phenology ma be correlated with trait syndromes. For instance, early phenologies may be associated with a lower investment in structural support---they may thus act as somewhat ruderal species with less competitive, but quicker to activate, hydraulic systems \citep{Lechowicz:1984cr} and higher growth rates \citep{Sun:2011eu,hahn2012}, though not all studies have found such correlations \citep[e.g.,][]{Craine:2012kl}.\\

\noindent \emph{How does competition with soil microbes affect phenology?}\\
To date most work on plant phenology has focused on how phenology ties to climate, or plant competition \citep[including for pollinators etc.,][]{Brody:1997ro}, yet it should also fundamentally be tied to soil microbial communities. Indeed, for most early species ties with climate may also, effectively, reflect ties to temporal microbial community dynamics, as soil microbial communities often turnover each spring producing a race for available nutrients between plants and microbes---the so called vernal damn \citep{Zak:1990ar,Lipson:2004es,Bardgett:2005ls}. While work integrating phenology and microbial communities is thin \citep[but see][]{dickens2013,esch2013}, recent research highlights that phenology can be shifted several days by microbial communities \citep{lau2012}. Further, extensive work in invasion biology highlights how soil microbes may play a critical role in promoting plant invasions \citep{Klironomos:2002jg,Wolfe:2005yr} suggesting the intersection of invasions, phenology and microbial effects may provide critical insights to the actual mechanisms by which phenology could affect plant invasions.

THIS SHOULD GO SOMEWHERE generally (since Cook et al. PNAS shows significant vernalization cues throughout the summer): Any such efforts will also probably need to extend back through the growing season -- need to extend  should be noted that for many species, especially perennial species, winter temperatures can also have a dramatic effect on spring phenology


Set up each as: (1) these are the forces and considerations (a-b: try to sort by relative importance), (2) this is what happens with climate change, this means this for invasions (argh, maybe just weave this in or see how it goes?), (3) evidence for this is X, Y, Z. Testable stuff is . .  . .\\
\\
\noindent \emph{Mid-season species}\\
- lower stress and disturbance due to physical environment\\
- but now stress can come from high competition with plants and microbes probably\\

- easier access to pollinators\\
- higher herbivory?\\
- mid-season droughts --> high stress (microbes do what?)\\

- * climate change may open up niches -- some non-stationarity, harder to predict across systems because highly dependent on precip, which is a bitch to predict, and maybe humidity matters or at least drought periods, also hard to predict.\\

- current work shows mid-season is way too stressful for invaders.

\noindent \emph{Late-season species}\\
- high risk of disturbance (rarely stress, things seem to just get killed)\\

- lowering competition\\
- microbial turnover\\

- * I think climatically this has shifted less (ASK Ben) but many plants show conservative strategies so may be big opportunity? \\

- current work shows some evidence of this in North temperate systems

\noindent \emph{Special considerations in systems with bimodal or year-round growing seasons}\\
- Whether there is variability in timing of start of season and magnitude. . .  . understanding soil moisture instead of just precip will be key.\\
- Tropics biotic forces may dominate (seed predation, disease, pollinators, seed dispersers) but small changes in abiotic forces may have large effects. .  . .

- * Some systems critically controlled by large-scale, longer-term weather patterns that may be key to predictions: PDO stuff (Julio's work).

* just means more about climate change, as compared to other stuff about what shapes plant strategies\\


\\
\\
Next sections:\\
Links from plant strategies to traits and ecosystems?\\
evolutionary history (this actually fits with mentioning traits!)\\
See below 'Ways froward' and onward.

\begin{enumerate}
\item Opening
\begin{enumerate}
\item Ditty about how much phenology has come into light with plant invasion research recently
\item Good, but need to focus on questions that integrate into a fundamental framework in order to make more useful, rapid progress
\item Point out Grime's theory here? see Table 2 in Grime 1977
\item Frame paper (`our goal . . . ') as pulling phenology away from climate change descriptions and back to basic life history theory, where it has spent much of its life before the rise of the IPCC in the last 15-20 years (don't say that of course, at least not like that last clause)
\item Big topics will be: reviewing how phenology fits into basic plant competition theory, reviewing recent literature (or not, maybe more reviewing how phenology fits into basic plant competition theory, pointing out along the way how recent literature may fit into this), suggesting progress through:
\begin{enumerate} 
\item competition vs. stress with a . . .
\item traits perspective (efficiences and trade-offs?)
\item microbial-plant perspective 
\item a dash of evolution?
\end{enumerate} 
\end{enumerate}
\includegraphics[]{../../figures/invasion_niches.png}
\item Back to basics: what are we dealing with here? `Principles architects of plant life history theory are: competition, predation, disturbance and disease (following Darwin). Others tend to focus on environmental conditions.' Paraphrased from Stanton et al. 2000. This is crazy contrast though because the environmental conditions (weather etc.) are a `disturbance' of sorts.
\begin{enumerate}
\item plants work on: 
\begin{enumerate}
\item aquisition
\item allocation
\item loss
\item so plants deal with competition and stress (from disturbance, herbivory, environmental variabiity and low productivity)
\end{enumerate}
\item try to view each of these through phenological lens - early vs. competitive strategies for aquisition, allocation and preformed buds or such, and avoiding loss by starting late, but then there's herbivory
\item uh, stress avoidance versus stress tolerators? Are any natives really good stress tolerators? Evidence from Konza and Fargo for no, talk about mid-season troughs and how that reviewer is an idiot
\item Frame sub-headings as (1) early, (2) mid and (3) late season strategies? Go though and review what plants are balancing in different habitats (well, let's just do temperate mesic and grassland) and what current non-native species evidence suggests may be going on. Yeah, yeah, this sounds good -- do this!
\item Early: lots of evidence that early is better (Munguia Rosas et al meta-analysis, Stanton et al. 2000); Grime's ruderal species here?
\item mid: stressful or competitive, but not a hotspot for invaders (if you ask me); Grime's competitive species
\item late: question for self and world: do any species start vegtative growth very late? Basically, when is the latest a species can start up? Seems like most get going by/before mid-summer . . . resources are too scarce after that?
\end{enumerate}
\item Ways forward
\begin{enumerate}
\item open-minded perspective on how plants are succeeding in novel habitats: balancing aquisition, allocation and loss (this could just be a header section before we lay out 3 subsections)
\item focus on plant-microbial competition as well as plant-plant competition
\item Think about correlated traits: Bolgren and Cowan 2008; call for thoughtful trait models, not just throwing everything in and seeing what comes out (this could move up to right under header then, as plant strategy theory should tell you what traits to think about).
\item (3 Aug 2013: This idea eems too half-baked for now, save for future, methinks.) Maybe think about evolution vs. plasticity and habitats under which phenology has evolved (Lechowicz 1984, ``relict requirements'', adaptive signal of phenology in Northern temperate forests may not always be obvious -- `evolutionary disequilibrium' -- but invaders don't seem to have that!)
\item Stationarity angle (goes with previous point somewhat)
\end{enumerate}
\item Closing
\begin{enumerate}
\item invasion biology and climate change research fields have suffered similar fates: prime for testing fundamental questions but burdened by emotive hypotheses. . . climate change ecology has general fallen into similar camps as macro- and microevolutionists: with those focusing on the large scale finding abundance evidence for a strong and deteministic role of climate in shaping species and communities, while local scale focus clearly shows how biotic interactions must critically structure stuff. (See Jablonski 2008.)
\end{enumerate}

\end{enumerate}
Yann's paper: Understory species are more temperature sensitive.\\
\\
Where does this stuff go? From Grime 1977 (note to self, I am not sure I agree with this shit at all, but interesting perspective):\\
- A well-known feature of many xerophytes and, in particular, the succulents is the rarity and erratic nature of flowing [sic].\\
- A characteristics of many shade-tolerant species is the paucity of flowering and seed production under heavily shaded conditions. This phenomenon is particularly obvious in British woodlands, where flowering in many common plants such as \emph{H. helix, Lonicera periclymenum,} and \emph{Rubus fruticosus} agg. is unsually restricted to plants and exposed sunlight at the margins of woods beneath gaps in the tree canopy.\\
\\
{\bf Ask Elsa:} to do a quick sweep of invasions phenology lit and add any important new references.


\newpage
\linenumbers

See cut stuff from temporalchange\_2013.tex. Here's some of it:
\\
Testing hypotheses about trade-offs with long-term climate dynamics or other plant strategies (e.g., early-flowering as a ruderal but not competitively dominant strategy) represents one area where there is immediate need explicit theory\\
\\
We hypothesize, however, that few species will have simple or dominant photoperiod controls for start of season events\\
\\
While these varying approaches may have lead to historical legacy effects\\
\\
Most temperate plant species have survived several expansions and contractions of their ranges over past glaciation cycles (Figure 4) meaning the molecular architecture underlying their phenologies are flexible enough---either through phenotypic plasticity or genotypic shifts, or some combination of the two---to sustain shifting climate and evolving ranges.\\
\\
Would also be great to work in somewhere how multiple phenological cues may only be noticeable at the extremes: for example, photoperiod may control many temperature cues but until warming is extreme (or in a growth chamber you massively muck with the photoperiod cue) you would not be able to note it. Same for some soil moisture issues.\\
\\
cite Martie's paper AmNat 1984 paper: The Quaternary history of the temperate forests has been one of repeated disruptions by glaciations such that present forest communities must be seen as transient assemblages rather than stable sets of coevolved species.\\

